\documentclass[10pt,a4paper]{article}
\usepackage[utf8]{inputenc}
\usepackage[ngerman]{babel}
\usepackage{amsmath}
\usepackage{amsfonts}
\usepackage{amssymb}
\usepackage{mathtools}
\usepackage[left=2cm,right=2cm,top=2cm,bottom=2cm]{geometry}
\author{Leonard K\"oenig}
\title{d4}
\begin{document}
\section{Äquivalenzrelationen}
\paragraph{Definition:} $R\subset A\times A$ ist eine Äquivalenzrelation, wenn sie reflexiv, symmetrisch und transitiv ist.

\paragraph{Bsp.:}
\begin{itemize}
\item $A=\mathbb{N}$, $R\subset\mathbb{N}\times\mathbb{N}$ definiert durch
\[
a\,R\,b \xLeftrightarrow{\text{Def.}} a\bmod 5
\]
(dh. gleicher Rest beim Teilen)
\item Die logische Äquivalenz ist eine Äquivalenzrelation auf der Menge der Boole'schen Terme/Formeln.
\item Kongruenz von Dreiecken
\end{itemize}

\paragraph{Definition (Äquivalenzklassen):}$R$ ist \"Aquivalenzrelation \"uber eine Menge $A$, $a\in A$.\\
Die \"Aquivalenzklasse von a ist die Menge aller Elemente von $A$, die zu $a$ in Relation stehen:
\[
[a]_{R} =_{Def.} \{x \in A | x\,R\,a\}
\]

Beispiel:\\
Relation $\equiv$ : mod $5$, $A = \mathbb{N}$ $a=17$
\begin{align*}
&[17]_{\text{mod }5} = \{2,7,12,17,22,...\} \\
&[8]_{\text{mod }5} = \{3,8,13,18, \ldots\}
\end{align*}
Es gibt $5$ verschiedene \"Aquivalenzklassen bzgl. dieser Relation.

\paragraph{Lemma: } Ist $R$ eine \"Aquivalenzrelation auf oder \"uber einer Menge $A$, dann sind zwei \"Aquivalenzklassen $[a]_R$ und $[b]_R$ entweder gleich oder disjunkt, dh. $[a]_R \cap [b]_R = \emptyset$

Folgerung: Die Menge der (verschiedenen) \"Aquivalenzklassen einer \"Aquivalenzrelation $R$ \"ueber einer Menge $A$ bildet eine Partition derselben Menge.\\
Begündung:\\
Betrachte Menge der \"Aquivalenzklassen
\[
\{[a_1]_R,[a_2]_R,\ldots\}
\]
Ist eine Partition, weil:
\begin{itemize}
\item $[a_i]_R$ ist nicht leer, weil $R$ reflexiv und somit $a_i \in [a_i]_R$
\item $\bigcup_{i \in I} [a_i]_R = A$, weil jedes $a\in A$ in seiner \"Aquivalenzklasse liegt.
\item paarweise disjunkt folgt aus dem Lemma.
\end{itemize}
\paragraph{Definiton (Repr\"asentantensystem): }Ist $R$ eine Äquivalenzrelation, dann nennt man eine Auswahl von genau einem Element aus jeder Äquivalenzklasse ein Representantensystem von $R$.

Beispiel: 1) $\equiv$ mod $5$ : $\{10,21,2,16,39\}$


2) $A=R \sim \subseteq R \times R$
\[
x\sim y \Leftrightarrow x-y \in \mathbb{Z}
\]
z.B.: $2.45 \sim 7.45$ und $1.3 \sim -2.7$

$\sim$ ist eine Äquivalenzrelation, weil:
\begin{itemize}
\item reflektiv, weil $x-x = 0 \in \mathbb{Z}$ für alle $x\in\mathbb{R}$
\item symmetrisch, weil $x\sim y \Rightarrow x-y \in \mathbb{Z} \Rightarrow -(x-y) \in \mathbb{Z} \Rightarrow  (y-x) \in \mathbb{Z} \Rightarrow y\sim x$
\item transitiv, weil $x\sim y \land y\sim z \Rightarrow x-y \in \mathbb{Z} \land y-z \in \mathbb{Z} \Rightarrow (x-y)+(y-z)\in \mathbb{Z} \Rightarrow x-z \in \mathbb{Z} \Rightarrow x\sim z$
\end{itemize}

$[1.6]_{\sim} = \{ \ldots\, , -1.4, -0.4, 0.6, 1.6, 2.6,\, \ldots \}$\\
Standardrepresentantensystem ist Intervall $[0,1)$.


\paragraph{Beispiel (Würfelspiel): }
\begin{itemize}
\item ein roter und ein blauer Würfel mit jeweils $6$ Seiten
\item Relation ist die Summe
\item Wir wollen eine Äquivalenzklasse, sodass wir alle mit den gleichen Summen zusammenfassen
\end{itemize}

\newpage
\paragraph{Übungen:}
\begin{enumerate}
\item $|A|=n$, $|B|=m$\\
Wie viele Relationen gibt es zwischen A und B?
\ \\
$R\subseteq A \times B$ Teilmenge $|A \times B| = m \cdot n$\\
Menge der Relationen $\mathcal{P}(A\times B)$, $|\mathcal{P}(A\times B)|=2^{n\cdot m}$
\item $|A|=2$\\
Wie viele Äquivalenzrelationen gibt es über A?
\ \\
$2$, nämlich die identische Relation und die Allrelation.
\item $|A|=3$\\
Wie viele Äquivalenzrelationen gibt es über A?
\ \\
Fallunterscheidung:
\begin{itemize}
\item mit $1$ Äquivalenzklasse: $1$ (Allrelation)
\item mit $2$ Äquivalenzklassen: $3$
\item mit $3$ Äquivalenzklassen: $1$ ($Id_A$)
\item[$\Rightarrow$] $5$
\end{itemize}
\item $|A|=4$\\
Wie viele Äquivalenzrelationen gibt es über A?
\ \\
Fallunterscheidung:
\begin{itemize}
\item mit $1$ Äquivalenzklasse: $1$ (Allrelation)
\item mit $2$ Äquivalenzklassen: $4+3$
\item mit $3$ Äquivalenzklassen: $6$
\item mit $4$ Äquivalenzklassen: $1$ ($Id_A$)
\item[$\Rightarrow$] $15$
\end{itemize}
\end{enumerate}

\section{Ordnungsrelationen}
\paragraph{Definition:}Eine (partielle) Ordnungsrelation ist eine Relation $R$ über einer Grundmenge $A$, die $3$ Eigenschaften hat:
\begin{enumerate}
\item reflexiv,
\item transitiv und
\item \emph{anti}symmetrisch.
\end{enumerate}
$(A,R)$ wird auch POSet (Partially Ordered Set) genannt.\\
Ist $R$ \emph{partielle} Ordnungsrelation über $A$ und $a,b \in A$ zwei Elemente von $A$, dann nennt man $a$ und $b$ vergleichbar, wenn $a\, R\, b$ oder $b\, R\, a$.\\
$R$ ist \emph{total}, wenn zwei \emph{beliebige} $a,b\in A$ vergleichbar sind.

\paragraph{Beispiele:}
\begin{itemize}
\item $\mathbb{N,Q,R}$ und $\leq$ oder mit $\geq$
\item $\mathbb{N}^+$ und $|$ (Teilbarkeit)
\item $\mathcal{P}(M)$ mit $\subseteq$ (Inklusionsrelation)
\item Menge von Wörtern $A$ und lexikographische Ordnung
\end{itemize}
Die Relationen $\leq$ und $\geq$ sowie die lexikographische Ordnung sind total; die Relationen $|$ und $\subseteq$ sind \emph{nicht} total.

\paragraph{Hasse-Diagramme}als minimalistische Beschreibung von endlichen POSets:\\
Beispiel: $\leq$-Relation über $\{1,2,3,4\}$ als gerichteter Graph
%TODO
% Grafik einfuegen

\paragraph{Definition:} Eine Relation $S$ über $A$ ist eine strenge (strikte) Ordnungsrelation, wenn sie transitiv und asymmetrisch ist.

\paragraph{Beispiele:}$<,>,\subseteq$, \glqq echter Teiler\grqq , \glqq echte Inklusion\grqq
\ \\
$R$ Ordnungsrelation\\
$S=R\setminus Id_A$ strenge Ordnungsrelation

\section{Funktionen}
\paragraph{Beschreibung:}Unter einer Funktion bzw. Abbildung (hier synonym) $f$ von einer Menge $A$ in eine Menge $B$ verstehen wir eine Zuordnung bei der jedem Element $a\in A$ ein eindeutig bestimmtes Element $b\in B$ entspricht.
\paragraph{Definition:}Ist $f$ eine Relation zwischen $A$ und $B$, sodass es für jedes $a\in A$ genau ein $b\in B$ gibt mit $a\, f\, b$ (oder $(a,b)\in f$).\\
In diesem Fall kann $f$ als Funktion schreiben:
\[
f:A\rightarrow B \qquad f(a)=b
\]
Hier ist $A$ der \emph{Definitionsbereich} und $B$ der \emph{Wertebereich}.
\[
\text{Bild von M:} \quad M\subseteq A \quad f(M) = \{f(x) | x\in M\}
\]
\[
\text{Bild von $f$:} \quad f(A)
\]
\[
\text{Urbild von $N$ unter $f$:} \quad \mathbb{N}\subseteq B \quad f^{-1}(A)
\]
\subsection{Eigenschaften von Funktionen}
Eine Funktion $f$ mit Definitionsbereich $A$ und Wertebereich $B$ ($f:A\rightarrow B$) ist
\begin{itemize}
\item \textbf{injektiv}, wenn verschiedenen Argumenten immer verschiedene Werte zugeordnet werden, dh.
\[
\forall_{a,a'\in A}: a\neq a' \Rightarrow f(a)\neq f(a')
\]
\item \textbf{surjektiv}, wenn jedes Element aus $B$ Bild eines $a$ aus $A$ ist, dh.
\[
\forall_{b\in B}\exists_{a\in A}: f(a)=b \xLeftrightarrow{\text{bzw.}} f(A)=f(B) \xLeftrightarrow{\text{bzw.}} \{b\in B | \exists_{a\in A}: f(a)=b\}=B
\]
\item \textbf{bijektiv}, wenn $f$ injektiv und surjektiv ist
\end{itemize}

\paragraph{Beispiel:}$f(x)=x^2$\\
$f$ als Funktion $f:\mathbb{R}\rightarrow\mathbb{R}$ ist
\begin{itemize}
\item[-] nicht injektiv (da bspw. $f(2)=f(-2)$)
\item[-] nicht surjektiv (da $-1\notin f(B)$)
\item[$\Rightarrow$] nicht bijektiv
\end{itemize}
$f$ als Funktion $f:\mathbb{R}\rightarrow\mathbb{R}^{\geq 0}$ ist
\begin{itemize}
\item[-] nicht injektiv (da bspw. $f(2)=f(-2)$)
\item[+] surjektiv
\item[$\Rightarrow$] nicht bijektiv
\end{itemize}
$f$ als Funktion $f:\mathbb{R}^{\geq 0}\rightarrow\mathbb{R}^{\geq 0}$ ist
\begin{itemize}
\item[+] injektiv
\item[+] surjektiv
\item[$\Rightarrow$] bijektiv $\Rightarrow f^{*}(f(x))=x$
\end{itemize}

\subsection{Operationen auf Funktionen}
\paragraph{Definition (Komposition von Funktionen):}Sind $f:A\rightarrow B$ und $g:B\rightarrow C$ Funktionen, dann ist die \textbf{Relation} $f\circ g$ eine Funktion von $A\rightarrow C$.\\
Diese Funktion wird mit $gf$ bezeichnet und es gilt
\[
(gf)(a)=g(f(a))
\]
%TODO
\textbf{GRAFIK EINFUEGEN!!!}
\paragraph{Definition (Umkehrfunktion):}Ist $f:A\rightarrow B$ eine bijektive Funktion, dann ist die inverse Relation $F^{-1}\subseteq B\times A$ eine Funktion, welche die Umkehrfunktion von $f$ genannt wird.
%TODO
% Grafik: Wann ist eine inverse Relation eine Funktion
\textbf{GRAFIK EINFUEGEN!!!}

\paragraph{Satz:}Für Funktionen $f:A\rightarrow B$ und $g:\rightarrow C$ gilt:
\begin{itemize}
\item $(f'f)(a)=a$
\item $(f'f)=(ff')=Id_A$
\end{itemize}
\paragraph{andere inverse Relationen als Funktionen}
\begin{itemize}
\item Ist $f$ injektiv, dann gibt es eine Funktion $h:B\rightarrow A$, sodass $(hf)=Id_A$\\
Dies ist im strengeren Sinne nicht unbedingt eine Umkehrfunktion, da $f$ nicht unbedingt bijektiv ist. Dadurch gibt es Elemente aus $B$ die für kein $a\in A$ mit $f(A)$ Bild sind.
$h$ definieren wir so, dass wir solche Elemente beliebig auf $A$ Abbilden, sodass komplett $B$ als Definitionsbereich genutzt werden kann, sprich $h(B)\in A$.
\item Ist $f$ surjektiv, dann gibt es eine Funktion $h:B\rightarrow A$, sodass $(fh)=Id_B$\\
Auch in diesem Fall ist $h$ nicht Umkehrfunktion von $f$. Es kann nämlich sein, dass es eine Menge $X\subseteq A$ (mit mehr als einem Element) gibt, sodass $F(X)=b$, mit $b\in B$.
Deshalb definieren wir unser $h$ so, dass nur ein Element $x$ Bild von einem solchen $b$ ist: $h(b)=x$ mit $x\in X$.
\item Ist $f$ injektiv und $g$ injektiv, dann ist auch $gf$ injektiv.
\item Ist $f$ und $g$ surjektiv, dann ist auch $gf$ surjektiv.
\item Ist $f$ und $g$ bijektiv, dann ist auch $gf$ bijektiv und
\[
(gf)^{-1}=f^{-1}g^{-1}
\]
\end{itemize}
\subsection{Äquivalenzrelationen}
\paragraph{Satz:}Sei $f:A\rightarrow B$ eine Funktion, dann ist die Relation $\sim_f\subseteq A\times A$ mit $a\sim_f a' \xLeftrightarrow{Def.} f(a)=f(a')$ eine Äquivalenzrelation über $A$.
\paragraph{Beispiel:}$f:\mathbb{N}\rightarrow\mathbb{N}$\\
$f(n)=(n \bmod 5)$: $\sim_f$ ist die Relation $\equiv \bmod 5$

\section{Exponential- und Logarithmusfunktionen}
\subsection{Die Eulersche Zahl $e$}
\begin{itemize}
\item $\lim\limits_{n \to \infty}\left(n+\frac{1}{n}\right)^n=e=2.718\ldots$
\item $\sum\limits_{k=0}^{\infty}\left(\frac{1}{k!}\right)=e=1+1+\frac{1}{2}+\frac{1}{3}+\frac{1}{24}+\ldots$
\item $\lim\limits_{n\to\infty}\left(1+\frac{x}{n}\right)^n=e^n$ für alle $x\in\mathbb{R}$
\item $\lim\limits_{n\to\infty}\left(1+\frac{x+y}{n}\right)^n=\lim\limits_{n\to\infty}\left(1+\frac{x}{n}\right)^n \cdot \lim\limits_{n\to\infty}\left(1+\frac{y}{n}\right)^n$
\end{itemize}
$exp: \mathbb{R}\to	\mathbb{R}^+$ bijektiv:
$e^x=y \Leftrightarrow \ln(y)=x$\\
%TODO
\textbf{GRAPH EINFUEGEN!!!}

\subsection{Eigenschaften}
Für beliebige Basen $a>0$:
\begin{itemize}
\item $a^x:\mathbb{R}\to\mathbb{R}^+$ bijektiv
\item Umkehrfunktion zu $a^x$ ist $\log_a:\mathbb{R}^+\to\mathbb{R}$
\item $\log(x)=\log_2(x)$ (auch manchmal $\text{ld}(x)$ \glq dualis\grq )
\item $\lg(x)=\log_{10}(x)$
\item $\ln(x)=\log_e(x)$ (\glq naturalis\grq )
\end{itemize}
Regeln:
\begin{itemize}
\item $\log_a(s\cdot t) = \log_a(s) + \log_a(t)$
\item $\log_a\left(\frac{1}{s}\right)=-\log_a(s)$
\end{itemize}

\section{Ganzzahlige Division}
\paragraph{Satz:}Für beliebige $a\in\mathbb{Z}$ und $d\in\mathbb{Z}^+$ gibt es eindeutige Werte $q,r\in\mathbb{Z}$ mit $a=qd+r$ mit $0\leq r<d$
\end{document}