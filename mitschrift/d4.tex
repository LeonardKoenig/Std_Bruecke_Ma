\documentclass[10pt,a4paper]{article}
\usepackage[utf8]{inputenc}
\usepackage[ngerman]{babel}
\usepackage{amsmath}
\usepackage{amsfonts}
\usepackage{amssymb}
\usepackage[left=2cm,right=2cm,top=2cm,bottom=2cm]{geometry}
\author{Leonard K\"oenig}
\title{d4}
\begin{document}
\section{Äquivalenzrelationen}
Def.: $R$ ist \"Aquivalenzrelation \"uber eine Menge $A$, $a\in A$.\\
Die \"Aquivalenzklasse von a ist die Menge aller Elemente von $A$, die zu $a$ in Relation stehen:
\[
[a]_{R} =_{Def.} \{x \in A | x\,R\,a\}
\]

Beispiel:\\
Relation $\equiv$ : mod $5$, $A = \mathbb{N}$ $a=17$
\begin{align*}
&[17]_{\text{mod }5} = \{2,7,12,17,22,...\} \\
&[8]_{\text{mod }5} = \{3,8,13,18, \ldots\}
\end{align*}
Es gibt $5$ verschiedene \"Aquivalenzklassen bzgl. dieser Relation.

\paragraph{Lemma: } Ist $R$ eine \"Aquivalenzrelation auf oder \"uber einer Menge $A$, dann sind zwei \"Aquivalenzklassen $[a]_R$ und $[b]_R$ entweder gleich oder disjunkt, dh. $[a]_R \cap [b]_R = \emptyset$

Folgerung: Die Menge der (verschiedenen) \"Aquivalenzklassen einer \"Aquivalenzrelation $R$ \"ueber einer Menge $A$ bildet eine Partition derselben Menge.\\
Begündung:\\
Betrachte Menge der \"Aquivalenzklassen
\[
\{[a_1]_R,[a_2]_R,\ldots\}
\]
Ist eine Partition, weil:
\begin{itemize}
\item $[a_i]_R$ ist nicht leer, weil $R$ reflexiv und somit $a_i \in [a_i]_R$
\item $\bigcup_{i \in I} [a_i]_R = A$, weil jedes $a\in A$ in seiner \"Aquivalenzklasse liegt.
\item paarweise disjunkt folgt aus dem Lemma.
\end{itemize}
\paragraph{Definiton (Repr\"asentantensystem): }Ist $R$ eine Äquivalenzrelation, dann nennt man eine Auswahl von genau einem Element aus jeder Äquivalenzklasse ein Representantensystem von $R$.

Beispiel: 1) $\equiv$ mod $5$ : $\{10,21,2,16,39\}$


2) $A=R \sim \subseteq R \times R$
\[
x\sim y \Leftrightarrow x-y \in \mathbb{Z}
\]
z.B.: $2.45 \sim 7.45$ und $1.3 \sim -2.7$

$\sim$ ist eine Äquivalenzrelation, weil:
\begin{itemize}
\item reflektiv, weil $x-x = 0 \in \mathbb{Z}$ für alle $x\in\mathbb{R}$
\item symmetrisch, weil $x\sim y \Rightarrow x-y \in \mathbb{Z} \Rightarrow -(x-y) \in \mathbb{Z} \Rightarrow  (y-x) \in \mathbb{Z} \Rightarrow y\sim x$
\item transitiv, weil $x\sim y \land y\sim z \Rightarrow x-y \in \mathbb{Z} \land y-z \in \mathbb{Z} \Rightarrow (x-y)+(y-z)\in \mathbb{Z} \Rightarrow x-z \in \mathbb{Z} \Rightarrow x\sim z$
\end{itemize}

$[1.6]_{\sim} = \{ \ldots\, , -1.4, -0.4, 0.6, 1.6, 2.6,\, \ldots \}$\\
Standardrepresentantensystem ist Intervall $[0,1)$.


\paragraph{Beispiel (Würfelspiel): }
\begin{itemize}
\item ein roter und ein blauer Würfel mit jeweils $6$ Seiten
\item Relation ist die Summe
\item Wir wollen eine Äquivalenzklasse, sodass wir alle mit den gleichen Summen zusammenfassen
\end{itemize}

\newpage
\paragraph{Übungen:}
\begin{enumerate}
\item $|A|=n$, $|B|=m$\\
Wie viele Relationen gibt es zwischen A und B?
\ \\
$R\subseteq A \times B$ Teilmenge $|A \times B| = m \cdot n$\\
Menge der Relationen $\mathcal{P}(A\times B)$, $|\mathcal{P}(A\times B)|=2^{n\cdot m}$
\item $|A|=2$\\
Wie viele Äquivalenzrelationen gibt es über A?
\ \\
$2$, nämlich die identische Relation und die Allrelation.
\item $|A|=3$\\
Wie viele Äquivalenzrelationen gibt es über A?
\ \\
Fallunterscheidung:
\begin{itemize}
\item mit $1$ Äquivalenzklasse: $1$ (Allrelation)
\item mit $2$ Äquivalenzklassen: $3$
\item mit $3$ Äquivalenzklassen: $1$ ($Id_A$)
\item[$\Rightarrow$] $5$
\end{itemize}
\item $|A|=4$\\
Wie viele Äquivalenzrelationen gibt es über A?
\ \\
Fallunterscheidung:
\begin{itemize}
\item mit $1$ Äquivalenzklasse: $1$ (Allrelation)
\item mit $2$ Äquivalenzklassen: $4+3$
\item mit $3$ Äquivalenzklassen: $6$
\item mit $4$ Äquivalenzklassen: $1$ ($Id_A$)
\item[$\Rightarrow$] $15$
\end{itemize}
\end{enumerate}

\section{Ordnungsrelationen}
\paragraph{Definition:}Eine (partielle) Ordnungsrelation ist eine Relation $R$ über einer Grundmenge $A$, die $3$ Eigenschaften hat:
\begin{enumerate}
\item reflexiv,
\item transitiv und
\item \emph{anti}symmetrisch.
\end{enumerate}
$(A,R)$ wird auch POSet (Partially Ordered Set) genannt.\\
Ist $R$ \emph{partielle} Ordnungsrelation über $A$ und $a,b \in A$ zwei Elemente von $A$, dann nennt man $a$ und $b$ vergleichbar, wenn $a\, R\, b$ oder $b\, R\, a$.\\
$R$ ist \emph{total}, wenn zwei \emph{beliebige} $a,b\in A$ vergleichbar sind.

\paragraph{Beispiele:}
\begin{itemize}
\item $\mathbb{N,Q,R}$ und $\leq$ oder mit $\geq$
\item $\mathbb{N}^+$ und $|$ (Teilbarkeit)
\item $\mathcal{P}(M)$ mit $\subseteq$ (Inklusionsrelation)
\item Menge von Wörtern $A$ und lexikographische Ordnung
\end{itemize}
Die Relationen $\leq$ und $\geq$ sowie die lexikographische Ordnung sind total; die Relationen $|$ und $\subseteq$ sind \emph{nicht} total.

\paragraph{Hasse-Diagramme}als minimalistische Beschreibung von endlichen POSets:\\
Beispiel: $\leq$-Relation über $\{1,2,3,4\}$ als gerichteter Graph
%TODO
% Grafik einfuegen

\paragraph{Definition:} Eine Relation $S$ über $A$ ist eine strenge (strikte) Ordnungsrelation, wenn sie transitiv und asymmetrisch ist.

\paragraph{Beispiele:}$<,>,\subseteq$, \glqq echter Teiler\grqq , \glqq echte Inklusion\grqq
\ \\
$R$ Ordnungsrelation\\
$S=R\setminus Id_A$ strenge Ordnungsrelation

\section{Funktionen}
\paragraph{Beschreibung:}Unter einer Funktion bzw. Abbildung (hier synonym) $f$ von einer Menge $A$ in eine Menge $B$ verstehen wir eine Zuordnung bei der jedem Element $a\in A$ ein eindeutig bestimmtes Element $b\in B$ entspricht.
\paragraph{Definition:}Ist $f$ eine Relation zwischen $A$ und $B$, sodass es für jedes $a\in A$ genau ein $b\in B$ gibt mit $a\, f\, b$ (oder $(a,b)\in f$).\\
In diesem Fall kann $f$ als Funktion schreiben:
\[
f:A\rightarrow B \qquad f(a)=b
\]
Hier ist $A$ der \emph{Definitionsbereich} und $B$ der \emph{Wertebereich}.
\[
\text{Bild von M:} \quad M\subseteq A \quad f(M) = \{f(x) | x\in M\}
\]
\[
\text{Bild von $f$:} \quad f(A)
\]
\[
\text{Urbild von $N$ unter $f$:} \quad \mathbb{N}\subseteq B \quad f^{-1}(A)
\]
\end{document}