%\documentclass[10pt,a4paper]{article}
%\usepackage[utf8]{inputenc}
%\usepackage[ngerman]{babel}
%\usepackage{amsmath}
%\usepackage{amsfonts}
%\usepackage{amssymb}
%\usepackage{amsthm}
%\usepackage{cancel}
%\usepackage{mathtools}
%\usepackage[left=2cm,right=2cm,top=2cm,bottom=2cm]{geometry}
%\author{Leonard K\"oenig}
%\title{titel}
%\begin{document}
\section{Pr\"adikatenlogik}
\paragraph{Quantoren}
\ \\$ \forall $ Allquantor \\
$ \exists $ Existenzquantor \\

\[
\forall_{x\in U}:P(x) \text{ w.A.},\text{ wenn }P(a) \text{ w.A. f\"ur alle } a \text{ aus } U
\]
\[
\exists_{x\in U}:P(x) \text{ w.A.},\text{ wenn ein }P(a) \text{ existiert, sodass } P(a) \text{ eine w.A.}
\]

\paragraph{Beispiel: Grenzwert-Definition}
\ \\
$(a_n)$ (mit $n\in \mathbb{N}$) hat Grenzwert $a$
\[
\Leftrightarrow \forall_{\varepsilon > 0} \, \exists_{n_0 \in \mathbb{N}} \, \forall_{n \geq n_0}: |a_n - a| < \varepsilon
\]

\paragraph{Definition (Pr\"adikat):}
\ \\
\glqq Ein logisches Pr\"adikat ist ein Ausdruck, der eine oder mehrere Variablen enth\"alt, sodass bei Belegung aller Variablen mit Objekten aus dem festgelegten Individuenbereich $B\setminus U$ eine Aussage entsteht.\grqq \\
\ \\
Beispiel:\\
\[
U=\mathbb{N}=\{0,1,2,3,\ldots\}
\]
$P(x)$ : \glq $x$ ist prim\grq \\
$R(x,y) = $\glq$x \leq y$\grq \\
$Q(x,y) = $\glq$x=y+2$\grq

\paragraph{\"Aquivalenzen bei Pr\"adikaten}
\begin{enumerate}
\item $\neg \forall_x : P(x) \equiv \exists_x: \neg P(x)$
\item $\neg \exists_x : P(x) \equiv \forall_x: \neg P(x)$
\item $\forall_x : P(x) \land \forall_x : Q(x) \equiv \forall : P(x) \land Q(x)$
\item $\exists_x : P(x) \lor \exists_x : Q(x) \equiv \exists_x : P(x) \lor Q(x)$
\item $\forall_{x,y} : R(x,y) \equiv \forall_{y,x} : R(x,y)$
\item $\exists_{x,y} : R(x,y) \equiv \exists_{y,x} : R(x,y)$
\item[$\perp$] $\forall_x : P(x) \lor \forall_x:Q(x) \equiv \forall_x : P(x) \lor Q(x)$
\item[$\perp$] $\exists_x : P(x) \land \exists : Q(x) \equiv \exists_x : P(x) \land Q(x)$
\item[$\perp$] $\forall_x \exists_y : R(x,y) \equiv \exists_y \forall_x : R(x,y)$
\end{enumerate}

\section{Beweistechniken}

\paragraph{Basiswissen:}
\ \\
\begin{itemize}
\item $\mathbb{N} = \{0,1,2,3,\ldots\}$ nat\"urliche Zahlen
\item $\mathbb{N}^{+} = \mathbb{N}\setminus\{0\}$
\item Addition und Multiplikation sind kommutativ, assoziativ und distributiv
\item $n\in\mathbb{N}$ ist durch $d\in\mathbb{N}^+$ teilbar (auch: \glqq d teilt u\grqq ), wenn ein $k\in\mathbb{N}$ existiert, sodass $n=k\cdot d$\\
Notation: $d|n$
\item $p\geq 2$ aus $\mathbb{N}$ ist prim, wenn sie nur durch $p$ und $1$ teilbar sind.
\item Jedes $n\geq 2$ besitzt eine \emph{eindeutige} Zerlegung in Primfaktoren, d.h.
\[
n = p_1 \cdot p_2 \cdot \ldots \cdot p_k
\]
mit $p_k$ Primfaktoren. Dies gilt insbesondere f\"ur $n=n$
\item $n\in\mathbb{N}$ ist gerade, wenn $n$ durch $2$ teilbar ist, d.h. $n=2k$ für ein $k\in\mathbb{N}$\\
$n$ ist ungerade, wenn ein solches k nicht existiert, bzw. $n=2k'+1$
\item $\forall n\in\mathbb{N},d\in\mathbb{N}^{+}$ gibt es eindeutige Werte $q,r\in\mathbb{N}$ mit $n=q\cdot d+r$ und $0\leq r< d$ mit $r=n\bmod d$
\end{itemize}

\subsection{Direkter Beweis}
Ein direkter Beweis hat die Form $p\rightarrow q$ mit Prämisse $p$ und Behauptung $q$. Man baut eine \glqq logische Kette\grqq , nach folgendem Schema:
\[
p=p_1\rightarrow p_2 \rightarrow p_3 \rightarrow \ldots \rightarrow p_k=q
\]
Wobei alle $p_n\rightarrow p_{n+1}$ möglichst triviale Folgerungen sind.

\paragraph{Beispiel:}Wenn $\underbrace{l|m \land m|n}_{p}$, dann $\underbrace{l|n}_{q}$.
\begin{proof}
\begin{align}
&l|m \land m|n \\
\Rightarrow &\exists_{k\in\mathbb{N}}: m=k\cdot l \land \exists_{j\in\mathbb{N}}: n=j\cdot m \\
\Rightarrow &\exists_{k,j\in\mathbb{N}}: m=k\cdot l \land n=j\cdot m \\
\Rightarrow &\exists_{k,j\in\mathbb{N}}: n=j(k\cdot l) \\
\Rightarrow &\exists_{k,j\in\mathbb{N}}: n=(j\cdot k)l && \mid k'=j\cdot k \\
\Rightarrow &\exists_{k,j\in\mathbb{N}}: n=k' \cdot l \\
\Rightarrow &l|n
\end{align}
\end{proof}

\subsection{Beweis durch Fallunterscheidung}
Ein Beweis der Form $p\rightarrow q \equiv \underbrace{(p\land r\rightarrow q)}_\text{Fall 1} \land \underbrace{(p\land\neg r \rightarrow q)}_\text{Fall 2}$ ist ein Beweis durch Fallunterscheidung, da wir die Fälle $r$ und $\neg r$ unterscheidet betrachten und für beide einen Teilbeweis führen.\\
Beispiel: sh. Beispiel zur Kontraposition.
\subsection{Indirekter Beweis}
\begin{itemize}
\item Kontra\emph{position} (Annahme der Gegenposition) \\
$p\rightarrow q \equiv \neg q \rightarrow\neg p$
\item Kontra\emph{diktion} (Widerspruch) \\
$p\rightarrow q \equiv p \land\neg q \rightarrow 0$
\end{itemize}

\paragraph{Beispiel}
$k|m\land k\nmid n \Rightarrow m\nmid n$
\paragraph{Beispiel (Widerspruch):}
\begin{proof}
\begin{align*}
&k|m\land k\nmid n \land m\mid n \\
\Rightarrow &k|m\land k\nmid n \land m\mid n && \mid k|m \land m|n \Rightarrow k|n \\
\Rightarrow &k|n \land k\nmid n \\
\Rightarrow &\bot
\end{align*}
\end{proof}

\paragraph{Beispiel (Kontraposition):}
\begin{proof}
\[
m|n \Rightarrow k\nmid m \lor k|n
\]
\begin{itemize}
\item Fall 1 ($k\nmid m$): w.A. nach $\neg p$
\item Fall 2 ($k\mid m$): $m|n \land k|m \Rightarrow k|n$, w.A. nach $\neg p$
\end{itemize}
\end{proof}

\subsection{Induktion}
\subsubsection{Einschub: Peano-Axiome der natürlichen Zahlen}
\begin{itemize}
\item $0\in\mathbb{N}$
\item fúr jedes $n\in\mathbb{N}$ gibt es einen \emph{eindeutigen} Nachfolger $s(n)\in\mathbb{N}$
\item verschiedene natürliche Zahlen haben verschiedene Nachfolger:\\
$n\neq n' \Rightarrow s(n) \neq s(n')$
\item $0$ ist kein Nachfolger
\item $\mathbb{N}$ ist die \emph{kleinste Menge, die diese Bedingungen erfüllt.}
\end{itemize}


\paragraph{Konsequenz: Rekursion}Man kann eine Funktion $f$ auf der Menge der natürlichen Zahlen definieren, indem man
\begin{itemize}
\item $f(0)$ festlegt und
\item $f(s(n))$ auf $f(n)$ zurückführt
\end{itemize}
Bspw.: $f:\mathbb{N}\rightarrow\{0,1\}$ mit
\begin{itemize}
\item $f(0)=0$
\item $f(s(n))=\neg f(n)$
\end{itemize}
%TODO
\textbf{TABELLE EINFUEGEN!!!}
% Auf Seite 14 oben

Insbesondere aber auch die Addition bzgl. $\mathbb{N}$:
\begin{align*}
f_n:\mathbb{N}\rightarrow\mathbb{N}\\
n\mapsto m+n
\end{align*}
\begin{enumerate}
\item $f_m(0)=m$
\item $f_m(s(n)) = s(f_m(n))$
\end{enumerate}

\paragraph{Am Beispiel $4+2$:}
\[
f_4(2)=f_4(s(1))=s(f_4(1))=s(f_4(s(0)))=s(s(f_4(0)))=s(s(4))=s(5)=6
\]
\subsubsection{Rekursion als Beweismethode: Induktion}
\paragraph{Idee:}Ein Prädikat $P(n)$ wobei $n\in\mathbb{N}$ ist wahr für alle $n\in\mathbb{N}$, wenn
\begin{enumerate}
\item $P(0)$ wahr ist und
\item für beliebige $n$ aus $P(n)$ $P(n+1)$ folgt (\glqq Dominoprinzip\grqq )
\end{enumerate}

(1) nennt man dabei \textbf{Induktionsanfang} oder \textbf{Induktionsverankerung} und (2) den \textbf{Induktionsschritt}, wobei man letzteren in die Gültigkeit von $P(n)$ als \textbf{Induktionsvoraussetzung} und die der \textbf{Induktionsbehauptung} $P(n+1)$ unterteilen kann.

\paragraph{Beispiel durch vollständige Induktion nach $n$:}$6|n^3 -n,\, \forall_{n\in\mathbb{N}}$\\
\begin{proof}
\ \\
\textbf{IA.:} $n=0$: $0^3-0=0$, w.A. \\
\textbf{IV.:} $6|n^3 -n$ w.A. für ein $n$ \\
\textbf{IB.:} aus $6|n^3 -n \rightarrow 6|(n-1)^3 -(n-1)$ \\
\textbf{IS.:}
\begin{align*}
&(n+1)^3 - (n+1) \\
= &n^3+3n^2+3n+1-(n+1) \\
= &(n^3-n) + 3n^2 + 3n && |k=\frac{n^3-n}{6} \text{ nach IV.}\\
= &6k +3(n^2+n) \\
= &6k + 3(n(n+1)) && |\text{Da $n$ oder $n+1$ gerade: } l=\frac{n(n+1)}{2} \\
= &6k+3\cdot 2l \\
= &6(k+l)
\end{align*}
\end{proof}

\section{Mengenlehre}
\paragraph{Beschreibung (G. Cantor):}\glqq Eine Menge ist die Zusammenfassung von bestimmten und wohl-strukturierten Objekten unserer Anschauung oder unseres Denkens
(die Elemente der Menge) zu einem Ganzen (der Menge selber).\grqq\\
Notation:
\begin{itemize}
\item $a$ gehört zur Menge $M$: $a\in M$
\item $a$ gehört \emph{nicht} zur Menge $M$: $a\notin M\equiv \neg(a\in M)$
\end{itemize}
Zwei Mengen $A$ und $B$ sind gleich, wenn sie sich aus den gleichen Elementen zusammensetzen, dh.
\[
a\in A \Leftrightarrow a\in B
\]
\subsection{Darstellung von Mengen}
\begin{enumerate}
\item Auflistung: $M=\{1,3,9,12\}=\{3,12,3,9,9,1\}$
\begin{itemize}
\item Reihenfolge und Wiederholung egal
\item \glq $\ldots$\grq -Notation:
\begin{itemize}
\item $\{1,2,3,\ldots\}=\mathbb{N}^+$
\item $\{0,2,4,6,\ldots\}$ gerade Zahlen
\end{itemize}
\end{itemize}
\item Abstraktionsprinzip (Z-F-Notation):\\
Menge wird über Grundbereich $U$ und Prädikat $P(x)$ definiert
\[
M=\{x\in U | P(x)\}
\]
(die Menge aller Objekte $a\in U$ für die $P(x)$ gilt.
\begin{itemize}
\item Bsp:
\begin{itemize}
\item $A=\{x\in\mathbb{N}|3|x \land x\leq 11\} = \{0,3,6,9\}$
\item $B=\{x\in\mathbb{N}|\forall_{j_\in\mathbb{N}^+}:(y|x\rightarrow y=x \lor y=1)\land x\geq 2\}=\{2,3,5,7,\ldots \}$ Menge der Primzahlen
\end{itemize}
\end{itemize}
\item Anschauliche Darstellung durch Venn-Diagramme:\\
%TODO
\textbf{VENN-DIAGRAMME EINFUEGEN!!!}
\end{enumerate}

\paragraph{Anmerkung:} Ein Element kann nur einfach in einer Menge vorkommen, sofern nicht als \glqq Multimenge\grqq deklariert:
\[
\text{Multimenge }\{1,5,3,3,3,1\} \neq \text{Multimenge }\{1,5,3,8\}
\]
\paragraph{Definition:}Eine Menge $A$ ist die Untermenge/Teilmenge von der Menge $B$, wenn $a\in A\rightarrow a\in B$ (auch $A\subseteq B$).\\

Beobachtung:
\begin{itemize}
\item $A=B \Leftrightarrow A\subseteq B \land B\subseteq A$
\item $A\subseteq B \land B\subseteq C \Rightarrow A\subseteq C$
\end{itemize}

Wichtige Grundmengen:
\begin{itemize}
\item $\mathbb{N}$: natürliche Zahlen
\item $\mathbb{Z}$: ganze Zahlen
\item $\mathbb{Q}$: rationale Zahlen (Quotienten)
\item $\mathbb{R}$: reelle Zahlen
\item $\mathbb{C}$: komplexe Zahlen
\end{itemize}
Es gilt:
\[
\mathbb{N}\subset\mathbb{Z}\subset\mathbb{Q}\subset\mathbb{R}\subset\mathbb{C}
\]
%TODO
\textbf{BEISPIELE}

\subsection{Operationen auf Mengen}
Sei $U$ ein vereinbartes Universum mit Teilmengen $A$ und $B$, dann ist
\begin{itemize}
\item die \textbf{Vereinigung} die Menge aller Objekte aus U, die in $A$ \emph{ODER} in $B$ liegen:\\
$a\in (A\cup B) \Leftrightarrow (a\in A) \lor (a\in B)$
\item der \textbf{Durchschnitt} die Menge aller Objekte die in $A$ \emph{UND} $B$ liegen ($A\cap B$)
\item die \textbf{Differenz} $A\setminus B$ die Menge aller Objekte, die in $A$ aber nicht in $B$ liegen
\item das \textbf{Komplementär} $\bar{A}$, die Menge aller Objekte aus $U$, die \emph{NICHT} in $A$ liegen ($\bar{A} = U\setminus A$)
\item die \textbf{symmetrische Differenz} $A\oplus B$ (oder $\div$) die Menge aller Objekte, die \emph{ENTWEDER} in $A$ \emph{ODER} in $B$ liegen:
\[
A\oplus B = (A\cup B)\setminus (A\cap B)
\]
\item die \textbf{leere Menge} $\emptyset = \{\,\}$
\end{itemize}

Zusammenhang Menge - Logik:
%TODO
\textbf{TABELLE EINFUEGEN!!!}

\subsection{Indexmengen}
\paragraph{Definition}Ist $I$ eine Menge und ist für jedes $i\in I$ eine Menge $A$ (in einem Universum U) gegeben, dann bezeichnet man
$\mathcal{A}=\{A_i|i\in I\}$ als eine \textbf{Mengenfamilie} der \textbf{Indexmenge} $I$.\\
Die Vereinigung und der Durchschnitt der Mengenfamilie $\mathcal{A}$ sind wie folgt definiert:
\begin{align*}
\bigcup_{i\in I} A_i = \{x\in U | \exists_{x\in I}: x\in A_i\}=\mathcal{A} \\
\bigcap_{i\in I} A_i = \{x\in U | \forall_{x\in U}: x\in A_i\} =\emptyset
\end{align*}

\paragraph{Beispiel:} $U=\mathbb{R}^2$, $I=\mathbb{N}^+$, $n\in\mathbb{N}$, $A_n$ Punkte in der Kreisscheibe mit Radius $n$ und dem Mittelpunkt $M(0,n)$
\begin{align*}
&\mathcal{A}=\{A_n|n\in \mathbb{N}^+\} \\
&\bigcap_{n\in\mathbb{N}}A_n=A_1 \\
&\bigcup_{n\in\mathbb{N}}A_n=\{ (x,y)\in\mathbb{R}^2 | y>0 \lor x=y=0\}
\end{align*}
%TODO
\textbf{GRAFIK EINFUEGEN!!!}
\subsection{Begriffe}
\paragraph{Partition:}Eine Mengenfamilie $\mathcal{A}=\{A_i | i\in I\}$, nennt man Partition/Zerlegung der Menge $A$, wenn
\begin{enumerate}
\item $A_i \neq \emptyset$
\item $\bigcup_{n\in\mathbb{N}}A_i=A$
\item $\bigcap_{n\in\mathbb{N}}A_i=\emptyset \xLeftrightarrow{\text{bzw.}} \forall_{i,j\in I,i\neq j}:A_i\cap A_j=\emptyset$
\end{enumerate}
\paragraph{Definition (Potenzmenge):}Für jede Menge $M$ wird die Menge aller Untermengen von $M$ als Potenzmenge $\mathcal{P(M)}$ bezeichnet.\\

Beispiele:
\begin{enumerate}
\item $M_0=\emptyset$ \\
$\mathcal{P}(M_0)=\{\emptyset\}$ \\
Anzahl der Elemente: $2^0=1$
\item $M_1=\{a\}$\\
$\mathcal{P}(M_1)=\{\emptyset,\{a\}\}$ \\
Anzahl der Elemente: $2^1=2$
\item $M_2=\{a,b\}$ \\
$\mathcal{P}(M_2)=\{\emptyset,\{a\},\{b\},\{a,b\}\}$ \\
Anzahl der Elemente: $2^2=4$
\end{enumerate}

\paragraph{Satz:}Die Potenzmenge einer endlichen Menge mit $n$ Elementen hat selbst $2^n$ Elemente.

\section{Relationen}
\paragraph{Definition (geordnetes Paar):}Ein geordnetes Paar $(a,b)$ von zwei Objekten $a$ und $b$ ist ein Konstrukt mit folgender Eigenschaft:
\[
(a,b) = (c,d) \xLeftrightarrow{\text{Def.}} a=c \land b=d
\]
\paragraph{Definition (kartesisches Produkt):}Das kartesische Produkt $A\times B$ von zwei Mengen $A$ und $B$ ist die Menge aller geordneten Paare $(a,b)$ mit $a\in A$ und $b\in B$, dh.
\[
A\times B = \{(a,b) | a\in A \land b\in B\}
\]

\paragraph{Beispiel:}$\{1,2\}\times \{2,3,4\}$
\[
\underbrace{ \{1,2\} }_{n\text{ Elemente}} \times \underbrace{ \{2,3,4\} }_{m\text{ Elemente}} = \underbrace{ \{ (1,2), (1,3), (1,4), (2,2), (2,3), (2,4) \} }_{n\cdot m\text{ Elemente}}
\]

\paragraph{Definition (Relation):}Eine Relation $R$ zwischen zwei Mengen $A$ und $B$ ist Teilmenge von $A\times B$.\\
Die Tatsache $(a,b)\in R$ kann auch durch $a\,R\,b$ ausgedrückt werden; gesprochen als \glqq a steht in Relation zu b\grqq .\\
Eine Teilmenge von $A\times A$ wird \glqq Relation auf/über $A$\grqq\ genannt.

Beispiele:
\begin{itemize}
\item $\emptyset \subseteq A\times B$ (leere Relation)
\item $A\times B \subseteq A\times B$ (Allrelation)
\item $Id_A = \{(a,a) | a\in A\} \subseteq A\times A$ (identische Relation)
\item Die \textbf{Teilbarkeitsrelation} $\mid$ kann als Relation über $\mathbb{N}$ oder $\mathbb{Z}$ betrachtet werden:
\[
\forall_{a,b\in\mathbb{N}}: a|b \Leftrightarrow \exists_{c\in\mathbb{N}}: a\cdot c = b
\]
\item die \textbf{Vergeleichsrelationen} $\leq$, $<$, $\geq$, $>$ ($=$ ist die identische Relation) sind Relationen über $\mathbb{N}$, $\mathbb{Z}$, $\mathbb{Q}$, $\mathbb{R}$
\item Jede Funktion $f:A\rightarrow B$ kann als Relation zwischen $A$ und $B$ betrachtet werden:
\[
(a,b)\in R \Leftrightarrow f(a)=b
\]
\end{itemize}

\subsection{Darstellung von Relationen}
\begin{itemize}
\item Als Menge von geordneten Paaren (nach Definition)
\item In Form von Tabellen ($A$: Spalten, $B$: Zeilen):
\begin{itemize}
\item Eintrag $1$ für $(a,b)\in R$
\item Eintrag $0$ für $(a,b)\not\in R$
\end{itemize}
\item Bipartiter Graph:\\
%TODO
\textbf{GRAPH EINFUEGEN!!!}
\item Bei Relationen über $A$: gerichteter Graph\\
%TODO
\textbf{GRAPH EINFUEGEN!!!}
\item Ein Diagramm\\
%TODO
\textbf{DIAGRAMM EINFUEGEN!!!}
\end{itemize}

\subsection{Relationsoperationen}
$R,S\subseteq A\times B$ Relation:
\begin{itemize}
\item \textbf{Durchschnittsrelation} $R\cap S$, \textbf{Vereinigungsrelation} $R\cup S$
\item \textbf{Komplementärrelation} $\bar{R}=(A\times B)\setminus R$
\item \textbf{Inverse Relation} $R^{-1}\subseteq B\times A$ definiert durch
\[
(b,a)\in R^{-1} \xLeftrightarrow{\text{Def.}} (a,b)\in R
\]
\item \textbf{Relationsverknüpfung/Komposition} $S\circ T\in A\times C$ mit $S\subseteq A\times B$ und $T\subseteq B\times C$ definiert durch
\[
(a,c)\in S\circ T \xLeftrightarrow{Def.}\exists_{b\in B}: (a,b)\in S \land (b,c)\in T
\]
Dh. $a\, (S\circ T)\, c \xLeftrightarrow{Def.} a\,S\,b \land b\,T\,c$
\end{itemize}

\paragraph{Beispiele:} $\leq,<,\geq,>\quad\in \mathbb{N}\times\mathbb{N}$
\begin{itemize}
\item $\leq\cup > = \mathbb{N}\times\mathbb{N}$
\item $\leq\cap\geq= Id_\mathbb{N}$ bzw. \glqq $(\leq \cap \geq) \,=\, (=)$\grqq 
\item $<\cap > = \emptyset$
\item $<\cup > = \bar{Id_\mathbb{N}} = \neq$
\item $\bar{\leq} = >$
\item $(\leq)^{-1} = \geq $
\item $\leq\circ\leq = \leq$
\item $< \circ < = \{(n,m) | n+2\leq m\}$
\end{itemize}

\subsection{Eigenschaften von Relationen}
$R\subseteq A\times A$ nennt man
\begin{itemize}
\item \textbf{reflexiv}, wenn $a\,R\,b$ gilt für alle $a\in A$, dh. $Id_A \subseteq R$
\item \textbf{symmetrisch}, wenn aus $a\,R\,b$ immer $b\,R\,c$ folgt, dh. $R=R^{-1}$
\item \textbf{transitiv}, wenn aus $a\,R\,b$ und $b\,R\,c$ immer $a\,R\,c$ folgt, dh. $R\circ R\subseteq R$
\item \textbf{antisymmetrisch}, wenn aus $a\,R\,b$ imd $b\,R\,a$ folgt, dass $a=b$, dh. $R\cap R^{-1}\subseteq Id_A$ (Bspw. $\leq$)
\item \textbf{assymetrisch}, wenn aus $a\,R\,b$ folgt, dass $(b,a)\notin R$, dh. $R\cap R^{-1}=\emptyset$ (Bspw. $<$)
\end{itemize}
%\end{document}