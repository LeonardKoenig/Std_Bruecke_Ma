%\documentclass[10pt,a4paper]{article}
%\usepackage[utf8]{inputenc}
%\usepackage[ngerman]{babel}
%\usepackage{amsmath}
%\usepackage{amsfonts}
%\usepackage{amssymb}
%\usepackage{mathtools}
%\usepackage[left=2cm,right=2cm,top=2cm,bottom=2cm]{geometry}
%\author{Leonard K\"oenig}
%\title{d4}
%\begin{document}

\section{Funktionen}
\paragraph{Beschreibung:}Unter einer Funktion bzw. Abbildung (hier synonym) $f$ von einer Menge $A$ in eine Menge $B$ verstehen wir eine Zuordnung bei der jedem Element $a\in A$ ein eindeutig bestimmtes Element $b\in B$ entspricht.
\paragraph{Definition:}Ist $f$ eine Relation zwischen $A$ und $B$, sodass es für jedes $a\in A$ genau ein $b\in B$ gibt mit $a\, f\, b$ (oder $(a,b)\in f$).\\
In diesem Fall kann $f$ als Funktion schreiben:
\[
f:A\rightarrow B \qquad f(a)=b
\]
Hier ist $A$ der \emph{Definitionsbereich} und $B$ der \emph{Wertebereich}.
\[
\text{Bild von M:} \quad M\subseteq A \quad f(M) = \{f(x) | x\in M\}
\]
\[
\text{Bild von $f$:} \quad f(A)
\]
\[
\text{Urbild von $N$ unter $f$:} \quad \mathbb{N}\subseteq B \quad f^{-1}(A)
\]
\subsection{Eigenschaften von Funktionen}
Eine Funktion $f$ mit Definitionsbereich $A$ und Wertebereich $B$ ($f:A\rightarrow B$) ist
\begin{itemize}
\item \textbf{injektiv}, wenn verschiedenen Argumenten immer verschiedene Werte zugeordnet werden, dh.
\[
\forall_{a,a'\in A}: a\neq a' \Rightarrow f(a)\neq f(a')
\]
\item \textbf{surjektiv}, wenn jedes Element aus $B$ Bild eines $a$ aus $A$ ist, dh.
\[
\forall_{b\in B}\exists_{a\in A}: f(a)=b \xLeftrightarrow{\text{bzw.}} f(A)=f(B) \xLeftrightarrow{\text{bzw.}} \{b\in B | \exists_{a\in A}: f(a)=b\}=B
\]
\item \textbf{bijektiv}, wenn $f$ injektiv und surjektiv ist
\end{itemize}

\paragraph{Beispiel:}$f(x)=x^2$\\
$f$ als Funktion $f:\mathbb{R}\rightarrow\mathbb{R}$ ist
\begin{itemize}
\item[-] nicht injektiv (da bspw. $f(2)=f(-2)$)
\item[-] nicht surjektiv (da $-1\notin f(B)$)
\item[$\Rightarrow$] nicht bijektiv
\end{itemize}
$f$ als Funktion $f:\mathbb{R}\rightarrow\mathbb{R}^{\geq 0}$ ist
\begin{itemize}
\item[-] nicht injektiv (da bspw. $f(2)=f(-2)$)
\item[+] surjektiv
\item[$\Rightarrow$] nicht bijektiv
\end{itemize}
$f$ als Funktion $f:\mathbb{R}^{\geq 0}\rightarrow\mathbb{R}^{\geq 0}$ ist
\begin{itemize}
\item[+] injektiv
\item[+] surjektiv
\item[$\Rightarrow$] bijektiv $\Rightarrow f^{*}(f(x))=x$
\end{itemize}

\subsubsection{Operationen auf Funktionen}
\paragraph{Definition (Komposition von Funktionen):}Sind $f:A\rightarrow B$ und $g:B\rightarrow C$ Funktionen, dann ist die \textbf{Relation} $f\circ g$ eine Funktion von $A\rightarrow C$.\\
Diese Funktion wird mit $gf$ bezeichnet und es gilt
\[
(gf)(a)=g(f(a))
\]
%TODO
\textbf{GRAFIK EINFUEGEN!!!}
\paragraph{Definition (Umkehrfunktion):}Ist $f:A\rightarrow B$ eine bijektive Funktion, dann ist die inverse Relation $F^{-1}\subseteq B\times A$ eine Funktion, welche die Umkehrfunktion von $f$ genannt wird.
%TODO
% Grafik: Wann ist eine inverse Relation eine Funktion
\textbf{GRAFIK EINFUEGEN!!!}

\paragraph{Satz:}Für Funktionen $f:A\rightarrow B$ und $g:\rightarrow C$ gilt:
\begin{itemize}
\item $(f'f)(a)=a$
\item $(f'f)=(ff')=Id_A$
\end{itemize}
\paragraph{andere inverse Relationen als Funktionen}
\begin{itemize}
\item Ist $f$ injektiv, dann gibt es eine Funktion $h:B\rightarrow A$, sodass $(hf)=Id_A$\\
Dies ist im strengeren Sinne nicht unbedingt eine Umkehrfunktion, da $f$ nicht unbedingt bijektiv ist. Dadurch gibt es Elemente aus $B$ die für kein $a\in A$ mit $f(A)$ Bild sind.
$h$ definieren wir so, dass wir solche Elemente beliebig auf $A$ Abbilden, sodass komplett $B$ als Definitionsbereich genutzt werden kann, sprich $h(B)\in A$.
\item Ist $f$ surjektiv, dann gibt es eine Funktion $h:B\rightarrow A$, sodass $(fh)=Id_B$\\
Auch in diesem Fall ist $h$ nicht Umkehrfunktion von $f$. Es kann nämlich sein, dass es eine Menge $X\subseteq A$ (mit mehr als einem Element) gibt, sodass $F(X)=b$, mit $b\in B$.
Deshalb definieren wir unser $h$ so, dass nur ein Element $x$ Bild von einem solchen $b$ ist: $h(b)=x$ mit $x\in X$.
\item Ist $f$ injektiv und $g$ injektiv, dann ist auch $gf$ injektiv.
\item Ist $f$ und $g$ surjektiv, dann ist auch $gf$ surjektiv.
\item Ist $f$ und $g$ bijektiv, dann ist auch $gf$ bijektiv und
\[
(gf)^{-1}=f^{-1}g^{-1}
\]
\end{itemize}

\paragraph{Satz:}Sei $f:A\rightarrow B$ eine Funktion, dann ist die Relation $\sim_f\subseteq A\times A$ mit $a\sim_f a' \xLeftrightarrow{Def.} f(a)=f(a')$ eine Äquivalenzrelation über $A$.
\paragraph{Beispiel:}$f:\mathbb{N}\rightarrow\mathbb{N}$\\
$f(n)=(n \bmod 5)$: $\sim_f$ ist die Relation $\equiv \bmod 5$

\subsection{Winkelfunktionen (Platzhalter)}
%TODO
% Subsection Titel aendern
\textbf{ERSTELLEN!!!}
\subsection{Exponential- und Logarithmusfunktionen}
\subsubsection{Die Eulersche Zahl $e$}
\begin{itemize}
\item $\lim\limits_{n \to \infty}\left(n+\frac{1}{n}\right)^n=e=2.718\ldots$
\item $\sum\limits_{k=0}^{\infty}\left(\frac{1}{k!}\right)=e=1+1+\frac{1}{2}+\frac{1}{3}+\frac{1}{24}+\ldots$
\item $\lim\limits_{n\to\infty}\left(1+\frac{x}{n}\right)^n=e^n$ für alle $x\in\mathbb{R}$
\item $\lim\limits_{n\to\infty}\left(1+\frac{x+y}{n}\right)^n=\lim\limits_{n\to\infty}\left(1+\frac{x}{n}\right)^n \cdot \lim\limits_{n\to\infty}\left(1+\frac{y}{n}\right)^n$
\end{itemize}
$exp: \mathbb{R}\to	\mathbb{R}^+$ bijektiv:
$e^x=y \Leftrightarrow \ln(y)=x$\\
%TODO
\textbf{GRAPH EINFUEGEN!!!}

\subsubsection{Eigenschaften}
Für beliebige Basen $a>0$:
\begin{itemize}
\item $a^x:\mathbb{R}\to\mathbb{R}^+$ bijektiv
\item Umkehrfunktion zu $a^x$ ist $\log_a:\mathbb{R}^+\to\mathbb{R}$
\item $\log(x)=\log_2(x)$ (auch manchmal $\text{ld}(x)$ \glq dualis\grq )
\item $\lg(x)=\log_{10}(x)$
\item $\ln(x)=\log_e(x)$ (\glq naturalis\grq )
\end{itemize}
Regeln:
\begin{itemize}
\item $\log_a(s\cdot t) = \log_a(s) + \log_a(t)$
\item $\log_a\left(\frac{1}{s}\right)=-\log_a(s)$
\end{itemize}

\section{Ganzzahlige Division}
\paragraph{Satz:}Für beliebige $a\in\mathbb{Z}$ und $d\in\mathbb{Z}^+$ gibt es eindeutige Werte $q,r\in\mathbb{Z}$ mit $a=qd+r$ mit $0\leq r<d$
%TODO
\\ \textbf{POLYNOM-TEIL ERSTELLEN!!!}
%\end{document}