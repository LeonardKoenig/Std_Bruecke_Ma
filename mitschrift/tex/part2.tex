%\documentclass[10pt,a4paper]{article}
%\usepackage[utf8]{inputenc}
%\usepackage[ngerman]{babel}
%\usepackage{amsmath}
%\usepackage{amsfonts}
%\usepackage{amssymb}
%\usepackage{amsthm}
%\usepackage{cancel}
%\usepackage{mathtools}
%\usepackage[left=2cm,right=2cm,top=2cm,bottom=2cm]{geometry}
%\author{Leonard K\"oenig}
%\title{titel}
%\begin{document}

\section{Mengenlehre}
\paragraph{Beschreibung (G. Cantor):}\glqq Eine Menge ist die Zusammenfassung von bestimmten und wohl-strukturierten Objekten unserer Anschauung oder unseres Denkens
(die Elemente der Menge) zu einem Ganzen (der Menge selber).\grqq\\
Notation:
\begin{itemize}
\item $a$ gehört zur Menge $M$: $a\in M$
\item $a$ gehört \emph{nicht} zur Menge $M$: $a\notin M\equiv \neg(a\in M)$
\end{itemize}
Zwei Mengen $A$ und $B$ sind gleich, wenn sie sich aus den gleichen Elementen zusammensetzen, dh.
\[
a\in A \Leftrightarrow a\in B
\]
\subsection{Darstellung von Operationen}
\subsubsection{Darstellung von Mengen}
\begin{enumerate}
\item Auflistung: $M=\{1,3,9,12\}=\{3,12,3,9,9,1\}$
\begin{itemize}
\item Reihenfolge und Wiederholung egal
\item \glq $\ldots$\grq -Notation:
\begin{itemize}
\item $\{1,2,3,\ldots\}=\mathbb{N}^+$
\item $\{0,2,4,6,\ldots\}$ gerade Zahlen
\end{itemize}
\end{itemize}
\item Abstraktionsprinzip (Z-F-Notation):\\
Menge wird über Grundbereich $U$ und Prädikat $P(x)$ definiert
\[
M=\{x\in U | P(x)\}
\]
(die Menge aller Objekte $a\in U$ für die $P(x)$ gilt.
\begin{itemize}
\item Bsp:
\begin{itemize}
\item $A=\{x\in\mathbb{N}|3|x \land x\leq 11\} = \{0,3,6,9\}$
\item $B=\{x\in\mathbb{N}|\forall_{j_\in\mathbb{N}^+}:(y|x\rightarrow y=x \lor y=1)\land x\geq 2\}=\{2,3,5,7,\ldots \}$ Menge der Primzahlen
\end{itemize}
\end{itemize}
\item Anschauliche Darstellung durch Venn-Diagramme:\\
%TODO
\textbf{VENN-DIAGRAMME EINFUEGEN!!!}
\end{enumerate}

\paragraph{Anmerkung:} Ein Element kann nur einfach in einer Menge vorkommen, sofern nicht als \glqq Multimenge\grqq deklariert:
\[
\text{Multimenge }\{1,5,3,3,3,1\} \neq \text{Multimenge }\{1,5,3,8\}
\]
\begin{defi}[Teilmenge]
Eine Menge $A$ ist die Untermenge/Teilmenge von der Menge $B$, wenn $a\in A\rightarrow a\in B$ (auch $A\subseteq B$).
\end{defi}

Beobachtung:
\begin{itemize}
\item $A=B \Leftrightarrow A\subseteq B \land B\subseteq A$
\item $A\subseteq B \land B\subseteq C \Rightarrow A\subseteq C$
\end{itemize}

Wichtige Grundmengen:
\begin{itemize}
\item $\mathbb{N}$: natürliche Zahlen
\item $\mathbb{Z}$: ganze Zahlen
\item $\mathbb{Q}$: rationale Zahlen (Quotienten)
\item $\mathbb{R}$: reelle Zahlen
\item $\mathbb{C}$: komplexe Zahlen
\end{itemize}
Es gilt:
\[
\mathbb{N}\subset\mathbb{Z}\subset\mathbb{Q}\subset\mathbb{R}\subset\mathbb{C}
\]
%TODO
\textbf{BEISPIELE}

\subsubsection{Operationen auf Mengen}
Sei $U$ ein vereinbartes Universum mit Teilmengen $A$ und $B$, dann ist
\begin{itemize}
\item die \textbf{Vereinigung} die Menge aller Objekte aus U, die in $A$ \emph{ODER} in $B$ liegen:\\
$a\in (A\cup B) \Leftrightarrow (a\in A) \lor (a\in B)$
\item der \textbf{Durchschnitt} die Menge aller Objekte die in $A$ \emph{UND} $B$ liegen ($A\cap B$)
\item die \textbf{Differenz} $A\setminus B$ die Menge aller Objekte, die in $A$ aber nicht in $B$ liegen
\item das \textbf{Komplementär} $\bar{A}$, die Menge aller Objekte aus $U$, die \emph{NICHT} in $A$ liegen ($\bar{A} = U\setminus A$)
\item die \textbf{symmetrische Differenz} $A\oplus B$ (oder $\div$) die Menge aller Objekte, die \emph{ENTWEDER} in $A$ \emph{ODER} in $B$ liegen:
\[
A\oplus B = (A\cup B)\setminus (A\cap B)
\]
\item die \textbf{leere Menge} $\emptyset = \{\,\}$
\end{itemize}

Zusammenhang Menge - Logik:
%TODO
\textbf{TABELLE EINFUEGEN!!!}

\subsubsection{Indexmengen}
\begin{defi}[Mengenfamilie und Indexmenge]
Ist $I$ eine Menge und ist für jedes $i\in I$ eine Menge $A$ (in einem Universum U) gegeben, dann bezeichnet man
$\mathcal{A}=\{A_i|i\in I\}$ als eine \textbf{Mengenfamilie} der \textbf{Indexmenge} $I$.\\
Die Vereinigung und der Durchschnitt der Mengenfamilie $\mathcal{A}$ sind wie folgt definiert:
\begin{align*}
\bigcup_{i\in I} A_i = \{x\in U | \exists_{x\in I}: x\in A_i\}=\mathcal{A} \\
\bigcap_{i\in I} A_i = \{x\in U | \forall_{x\in U}: x\in A_i\} =\emptyset
\end{align*}
\end{defi}

\paragraph{Beispiel:} $U=\mathbb{R}^2$, $I=\mathbb{N}^+$, $n\in\mathbb{N}$, $A_n$ Punkte in der Kreisscheibe mit Radius $n$ und dem Mittelpunkt $M(0,n)$
\begin{align*}
&\mathcal{A}=\{A_n|n\in \mathbb{N}^+\} \\
&\bigcap_{n\in\mathbb{N}}A_n=A_1 \\
&\bigcup_{n\in\mathbb{N}}A_n=\{ (x,y)\in\mathbb{R}^2 | y>0 \lor x=y=0\}
\end{align*}
%TODO
\textbf{GRAFIK EINFUEGEN!!!}
\subsubsection{Begriffe}
\paragraph{Partition:}Eine Mengenfamilie $\mathcal{A}=\{A_i | i\in I\}$, nennt man Partition/Zerlegung der Menge $A$, wenn
\begin{enumerate}
\item $A_i \neq \emptyset$
\item $\bigcup_{n\in\mathbb{N}}A_i=A$
\item $\bigcap_{n\in\mathbb{N}}A_i=\emptyset \xLeftrightarrow{\text{bzw.}} \forall_{i,j\in I,i\neq j}:A_i\cap A_j=\emptyset$
\end{enumerate}

\begin{defi}[Potenzmenge]
Für jede Menge $M$ wird die Menge aller Untermengen von $M$ als Potenzmenge $\mathcal{P(M)}$ bezeichnet.
\end{defi}

Beispiele:
\begin{enumerate}
\item $M_0=\emptyset$ \\
$\mathcal{P}(M_0)=\{\emptyset\}$ \\
Anzahl der Elemente: $2^0=1$
\item $M_1=\{a\}$\\
$\mathcal{P}(M_1)=\{\emptyset,\{a\}\}$ \\
Anzahl der Elemente: $2^1=2$
\item $M_2=\{a,b\}$ \\
$\mathcal{P}(M_2)=\{\emptyset,\{a\},\{b\},\{a,b\}\}$ \\
Anzahl der Elemente: $2^2=4$
\end{enumerate}

\paragraph{Satz:}Die Potenzmenge einer endlichen Menge mit $n$ Elementen hat selbst $2^n$ Elemente.

\subsection{Relationen}
\begin{defi}[geordnetes Paar]
Ein geordnetes Paar $(a,b)$ von zwei Objekten $a$ und $b$ ist ein Konstrukt mit folgender Eigenschaft:
\[
(a,b) = (c,d) \xLeftrightarrow{\text{Def.}} a=c \land b=d
\]
\end{defi}
\begin{defi}[kartesisches Produkt]
Das kartesische Produkt $A\times B$ von zwei Mengen $A$ und $B$ ist die Menge aller geordneten Paare $(a,b)$ mit $a\in A$ und $b\in B$, dh.
\[
A\times B = \{(a,b) | a\in A \land b\in B\}
\]
\end{defi}
%
\paragraph{Beispiel:}$\{1,2\}\times \{2,3,4\}$
\[
\underbrace{ \{1,2\} }_{n\text{ Elemente}} \times \underbrace{ \{2,3,4\} }_{m\text{ Elemente}} = \underbrace{ \{ (1,2), (1,3), (1,4), (2,2), (2,3), (2,4) \} }_{n\cdot m\text{ Elemente}}
\]

\begin{defi}[Relation]
Eine Relation $R$ zwischen zwei Mengen $A$ und $B$ ist Teilmenge von $A\times B$.\\
Die Tatsache $(a,b)\in R$ kann auch durch $a\,R\,b$ ausgedrückt werden; gesprochen als \glqq a steht in Relation zu b\grqq .\\
Eine Teilmenge von $A\times A$ wird \glqq Relation auf/über $A$\grqq\ genannt.
\end{defi}
%
\paragraph{Beispiele:}
\begin{itemize}
\item $\emptyset \subseteq A\times B$ (leere Relation)
\item $A\times B \subseteq A\times B$ (Allrelation)
\item $Id_A = \{(a,a) | a\in A\} \subseteq A\times A$ (identische Relation)
\item Die \textbf{Teilbarkeitsrelation} $\mid$ kann als Relation über $\mathbb{N}$ oder $\mathbb{Z}$ betrachtet werden:
\[
\forall_{a,b\in\mathbb{N}}: a|b \Leftrightarrow \exists_{c\in\mathbb{N}}: a\cdot c = b
\]
\item die \textbf{Vergeleichsrelationen} $\leq$, $<$, $\geq$, $>$ ($=$ ist die identische Relation) sind Relationen über $\mathbb{N}$, $\mathbb{Z}$, $\mathbb{Q}$, $\mathbb{R}$
\item Jede Funktion $f:A\rightarrow B$ kann als Relation zwischen $A$ und $B$ betrachtet werden:
\[
(a,b)\in R \Leftrightarrow f(a)=b
\]
\end{itemize}

\subsubsection{Darstellung von Relationen}
\begin{itemize}
\item Als Menge von geordneten Paaren (nach Definition)
\item In Form von Tabellen ($A$: Spalten, $B$: Zeilen):
\begin{itemize}
\item Eintrag $1$ für $(a,b)\in R$
\item Eintrag $0$ für $(a,b)\not\in R$
\end{itemize}
\item Bipartiter Graph:\\
%TODO
\textbf{GRAPH EINFUEGEN!!!}
\item Bei Relationen über $A$: gerichteter Graph\\
%TODO
\textbf{GRAPH EINFUEGEN!!!}
\item Ein Diagramm\\
%TODO
\textbf{DIAGRAMM EINFUEGEN!!!}
\end{itemize}

\subsubsection{Relationsoperationen}
$R,S\subseteq A\times B$ Relation:
\begin{itemize}
\item \textbf{Durchschnittsrelation} $R\cap S$, \textbf{Vereinigungsrelation} $R\cup S$
\item \textbf{Komplementärrelation} $\bar{R}=(A\times B)\setminus R$
\item \textbf{Inverse Relation} $R^{-1}\subseteq B\times A$ definiert durch
\[
(b,a)\in R^{-1} \xLeftrightarrow{\text{Def.}} (a,b)\in R
\]
\item \textbf{Relationsverknüpfung/Komposition} $S\circ T\in A\times C$ mit $S\subseteq A\times B$ und $T\subseteq B\times C$ definiert durch
\[
(a,c)\in S\circ T \xLeftrightarrow{Def.}\exists_{b\in B}: (a,b)\in S \land (b,c)\in T
\]
Dh. $a\, (S\circ T)\, c \xLeftrightarrow{Def.} a\,S\,b \land b\,T\,c$
\end{itemize}

\paragraph{Beispiele:} $\leq,<,\geq,>\quad\in \mathbb{N}\times\mathbb{N}$
\begin{itemize}
\item $\leq\cup > = \mathbb{N}\times\mathbb{N}$
\item $\leq\cap\geq= Id_\mathbb{N}$ bzw. \glqq $(\leq \cap \geq) \,=\, (=)$\grqq 
\item $<\cap > = \emptyset$
\item $<\cup > = \bar{Id_\mathbb{N}} = \neq$
\item $\bar{\leq} = >$
\item $(\leq)^{-1} = \geq $
\item $\leq\circ\leq = \leq$
\item $< \circ < = \{(n,m) | n+2\leq m\}$
\end{itemize}

\subsubsection{Eigenschaften von Relationen}
$R\subseteq A\times A$ nennt man
\begin{itemize}
\item \textbf{reflexiv}, wenn $a\,R\,b$ gilt für alle $a\in A$, dh. $Id_A \subseteq R$
\item \textbf{symmetrisch}, wenn aus $a\,R\,b$ immer $b\,R\,c$ folgt, dh. $R=R^{-1}$
\item \textbf{transitiv}, wenn aus $a\,R\,b$ und $b\,R\,c$ immer $a\,R\,c$ folgt, dh. $R\circ R\subseteq R$
\item \textbf{antisymmetrisch}, wenn aus $a\,R\,b$ imd $b\,R\,a$ folgt, dass $a=b$, dh. $R\cap R^{-1}\subseteq Id_A$ (Bspw. $\leq$)
\item \textbf{assymetrisch}, wenn aus $a\,R\,b$ folgt, dass $(b,a)\notin R$, dh. $R\cap R^{-1}=\emptyset$ (Bspw. $<$)
\end{itemize}

\subsection{Äquivalenzrelationen}
\begin{defi}[Äquivalenzrelation]
$R\subset A\times A$ ist eine Äquivalenzrelation, wenn sie reflexiv, symmetrisch und transitiv ist.
\end{defi}
%
\paragraph{Bsp.:}
\begin{itemize}
\item $A=\mathbb{N}$, $R\subset\mathbb{N}\times\mathbb{N}$ definiert durch
\[
a\,R\,b \xLeftrightarrow{\text{Def.}} a\bmod 5
\]
(dh. gleicher Rest beim Teilen)
\item Die logische Äquivalenz ist eine Äquivalenzrelation auf der Menge der Boole'schen Terme/Formeln.
\item Kongruenz von Dreiecken
\end{itemize}

\begin{defi}[Äquivalenzklassen]
$R$ ist \"Aquivalenzrelation \"uber eine Menge $A$, $a\in A$.\\
Die \"Aquivalenzklasse von a ist die Menge aller Elemente von $A$, die zu $a$ in Relation stehen:
\[
[a]_{R} =_{Def.} \{x \in A | x\,R\,a\}
\]
\end{defi}
%
\paragraph{Beispiel:}
Relation $\equiv$ : mod $5$, $A = \mathbb{N}$ $a=17$
\begin{align*}
&[17]_{\text{mod }5} = \{2,7,12,17,22,...\} \\
&[8]_{\text{mod }5} = \{3,8,13,18, \ldots\}
\end{align*}
Es gibt $5$ verschiedene \"Aquivalenzklassen bzgl. dieser Relation.

\paragraph{Lemma: } Ist $R$ eine \"Aquivalenzrelation auf oder \"uber einer Menge $A$, dann sind zwei \"Aquivalenzklassen $[a]_R$ und $[b]_R$ entweder gleich oder disjunkt, dh. $[a]_R \cap [b]_R = \emptyset$

Folgerung: Die Menge der (verschiedenen) \"Aquivalenzklassen einer \"Aquivalenzrelation $R$ \"ueber einer Menge $A$ bildet eine Partition derselben Menge.\\
Begündung:\\
Betrachte Menge der \"Aquivalenzklassen
\[
\{[a_1]_R,[a_2]_R,\ldots\}
\]
Ist eine Partition, weil:
\begin{itemize}
\item $[a_i]_R$ ist nicht leer, weil $R$ reflexiv und somit $a_i \in [a_i]_R$
\item $\bigcup_{i \in I} [a_i]_R = A$, weil jedes $a\in A$ in seiner \"Aquivalenzklasse liegt.
\item paarweise disjunkt folgt aus dem Lemma.
\end{itemize}
\paragraph{Definiton (Repr\"asentantensystem): }Ist $R$ eine Äquivalenzrelation, dann nennt man eine Auswahl von genau einem Element aus jeder Äquivalenzklasse ein Representantensystem von $R$.

Beispiel: 1) $\equiv$ mod $5$ : $\{10,21,2,16,39\}$


2) $A=R \sim \subseteq R \times R$
\[
x\sim y \Leftrightarrow x-y \in \mathbb{Z}
\]
z.B.: $2.45 \sim 7.45$ und $1.3 \sim -2.7$

$\sim$ ist eine Äquivalenzrelation, weil:
\begin{itemize}
\item reflektiv, weil $x-x = 0 \in \mathbb{Z}$ für alle $x\in\mathbb{R}$
\item symmetrisch, weil $x\sim y \Rightarrow x-y \in \mathbb{Z} \Rightarrow -(x-y) \in \mathbb{Z} \Rightarrow  (y-x) \in \mathbb{Z} \Rightarrow y\sim x$
\item transitiv, weil $x\sim y \land y\sim z \Rightarrow x-y \in \mathbb{Z} \land y-z \in \mathbb{Z} \Rightarrow (x-y)+(y-z)\in \mathbb{Z} \Rightarrow x-z \in \mathbb{Z} \Rightarrow x\sim z$
\end{itemize}

$[1.6]_{\sim} = \{ \ldots\, , -1.4, -0.4, 0.6, 1.6, 2.6,\, \ldots \}$\\
Standardrepresentantensystem ist Intervall $[0,1)$.


\paragraph{Beispiel (Würfelspiel): }
\begin{itemize}
\item ein roter und ein blauer Würfel mit jeweils $6$ Seiten
\item Relation ist die Summe
\item Wir wollen eine Äquivalenzklasse, sodass wir alle mit den gleichen Summen zusammenfassen
\end{itemize}

\newpage
\paragraph{Übungen:}
\begin{enumerate}
\item $|A|=n$, $|B|=m$\\
Wie viele Relationen gibt es zwischen A und B?
\ \\
$R\subseteq A \times B$ Teilmenge $|A \times B| = m \cdot n$\\
Menge der Relationen $\mathcal{P}(A\times B)$, $|\mathcal{P}(A\times B)|=2^{n\cdot m}$
\item $|A|=2$\\
Wie viele Äquivalenzrelationen gibt es über A?
\ \\
$2$, nämlich die identische Relation und die Allrelation.
\item $|A|=3$\\
Wie viele Äquivalenzrelationen gibt es über A?
\ \\
Fallunterscheidung:
\begin{itemize}
\item mit $1$ Äquivalenzklasse: $1$ (Allrelation)
\item mit $2$ Äquivalenzklassen: $3$
\item mit $3$ Äquivalenzklassen: $1$ ($Id_A$)
\item[$\Rightarrow$] $5$
\end{itemize}
\item $|A|=4$\\
Wie viele Äquivalenzrelationen gibt es über A?
\ \\
Fallunterscheidung:
\begin{itemize}
\item mit $1$ Äquivalenzklasse: $1$ (Allrelation)
\item mit $2$ Äquivalenzklassen: $4+3$
\item mit $3$ Äquivalenzklassen: $6$
\item mit $4$ Äquivalenzklassen: $1$ ($Id_A$)
\item[$\Rightarrow$] $15$
\end{itemize}
\end{enumerate}

\subsection{Ordnungsrelationen}
\begin{defi}[Ordnungsrelation]
Eine (partielle) Ordnungsrelation ist eine Relation $R$ über einer Grundmenge $A$, die $3$ Eigenschaften hat:
\begin{enumerate}
\item reflexiv,
\item transitiv und
\item \emph{anti}symmetrisch.
\end{enumerate}
$(A,R)$ wird auch POSet (Partially Ordered Set) genannt.\\
Ist $R$ \emph{partielle} Ordnungsrelation über $A$ und $a,b \in A$ zwei Elemente von $A$, dann nennt man $a$ und $b$ vergleichbar, wenn $a\, R\, b$ oder $b\, R\, a$.\\
$R$ ist \emph{total}, wenn zwei \emph{beliebige} $a,b\in A$ vergleichbar sind.
\end{defi}

\paragraph{Beispiele:}
\begin{itemize}
\item $\mathbb{N,Q,R}$ und $\leq$ oder mit $\geq$
\item $\mathbb{N}^+$ und $|$ (Teilbarkeit)
\item $\mathcal{P}(M)$ mit $\subseteq$ (Inklusionsrelation)
\item Menge von Wörtern $A$ und lexikographische Ordnung
\end{itemize}
Die Relationen $\leq$ und $\geq$ sowie die lexikographische Ordnung sind total; die Relationen $|$ und $\subseteq$ sind \emph{nicht} total.

\paragraph{Hasse-Diagramme}als minimalistische Beschreibung von endlichen POSets:\\
Beispiel: $\leq$-Relation über $\{1,2,3,4\}$ als gerichteter Graph
%TODO
% Grafik einfuegen

\begin{defi}[strikte Ordnungsrelation]
Eine Relation $S$ über $A$ ist eine strenge (strikte) Ordnungsrelation, wenn sie transitiv und asymmetrisch ist.
\end{defi}
%
\paragraph{Beispiele:}$<,>,\subseteq$, \glqq echter Teiler\grqq , \glqq echte Inklusion\grqq
\ \\
$R$ Ordnungsrelation\\
$S=R\setminus Id_A$ strenge Ordnungsrelation

%\end{document}