%\documentclass[10pt,a4paper]{article}
%\usepackage[utf8]{inputenc}
%\usepackage[ngerman]{babel}
%\usepackage{amsmath}
%\usepackage{amsfonts}
%\usepackage{amssymb}
%\usepackage{amsthm} % theorems
%\usepackage{cancel} % strikethrough
%\usepackage{tabto} % tabbing
%%\usepackage[left=2cm,right=2cm,top=2cm,bottom=2cm]{geometry}
%\author{Leonard König}
%\title{Mathe-Bruecke_d1}
%\date{2015-09-29}
%\begin{document}
\section{Aussagenlogik}
Aussage: (formalsprachliches) Gebilde (Satz) mit der Eigenschaft entweder wahr oder falsch zu sein.
%
\begin{itemize}
\item Zweiwertigkeitsprinzip
\item Extentionalitätsprinizip (Wahrheitswert einer zusammengesetzten Aussage ergibt sich aus denen der einzelnen Aussagen)
\end{itemize}
%
\subsection{Konkatenieren von Aussagen:}
\begin{itemize}
\item Negation (\( \neg \)) \tab{(unär)}
\item Konjunktion (\( \land \)) \tab{(binär)}
\item Disjunktion (\( \lor \)) \tab{(binär)}
\item Implikation (\( \rightarrow \)) \tab{(binär)}
\item Äquivalenz (\( \leftrightarrow \)) \tab{(binär)}
\item ( Antivalenz (\( \oplus \)) ) \tab{(binär)}
\end{itemize}
%
\subsection{Beispiele und Anwendung}
%
Werte für eine Aussage sind wahr (\( w,t,1 \)) und falsch (\( f,0 \)).
%
\subsubsection{Beispiele für Aussagen}
\begin{itemize}
\item \glqq $7$ ist prim\grqq\tab{$1$}
\item \glqq $9$ ist prim\grqq\tab{$0$}
\item \glqq $10$ ist gerade\grqq\tab{$1$}
\item \glqq $7$ ist prim\grqq\tab{$1$}
\item \glqq $7$ ist prim $\lor$ $7$ ist ungerade\grqq\tab{$1$}
\item \glqq $7$ ist prim $\oplus$ $7$ ist ungerade\grqq\tab{$0$}
\item \glqq $\sqrt{2}$ ist rational\grqq\tab{$0$}
\item \glqq Jede gerade Zahl $n\geq 4$\\
kann als Summe von zwei Primzahlen\\
dargestellt werden\grqq\tab{? (Goldbachsche Vermutung)}
\end{itemize}
%
Es muss also \emph{nicht} klar sein, ob die Aussage wahr oder falsch ist!
%
\subsection{Keine Aussagen}
\begin{itemize}
\item \glqq $x:$ $x$ ist falsch\grqq\tab{Paradoxon}
\item \glqq $n$ ist prim\grqq\tab{Prädikat. Sh. spätere Vorlesungen}
\end{itemize}
%
\textbf{WAHRHEITSTAFEL EINFUEGEN}
%TODO
\subsection{Syntaktische und Semantische Korrektheit}
\begin{itemize}
\item \( x + - y \)\tab{syntaktisch inkorrekt}
\item \( \frac{x}{y} \)\tab{syntaktisch korrekt \& semantisch korrekt für \( y \neq 0 \)}
\end{itemize}
%
\subsection{Definition}
\newtheorem{defi}{Definition}
\begin{defi}[Term der Aussagenlogik]
Gegeben Variablenmenge $V = \{ x,y,z,x_1,x_2, ...\} $
\begin{enumerate}
\item Jede Variable in $V$ und $0,1$ sind \glq Terme\grq\ bzw. Formeln der Aussagenlogik
\item falls $t$ ein Term, dann ist auch ($\neg t$) ein Term\tab{$\mid$ erhöht Rang um 1}
\item sind $s,t$ Terme der Aussagenlogik, \\
dann sind auch $(s \land t)$, $(s \lor t)$, $(s\rightarrow t)$, $(s\leftrightarrow t)$ \\
Terme der Aussagenlogik\tab{$\mid$ neuer Rang: Maximum der beiden \glq Subränge\grq\ $+1$}
\item Alle Terme der Aussagenlogik lassen sich wie beschrieben bilden.
\end{enumerate}
\end{defi}
%
Anmerkungen
\begin{itemize}
\item Wenn wir nur $\neg$, $\land$, $\lor$ verwenden, erzeugen wir die sogenannten \glq Booleschen Terme\grq
\item In der Aussagenlogik sind \glq Term\grq\ und \glq Formel\grq\ synonym.
\item Um auf \emph{Gleichheit} (beachte: \emph{nicht} Äquivalenz!) zu prüfen, können wir den Syntaxbaum nutzen (siehe Abb. ???)
%TODO Abbildungsreferenz
\\ \textbf{SYNTAXBAUM EINFUEGEN}
%TODO
\item Bindungsstärke (zur Klammersetzung):
\begin{itemize}
\item äußere Klammern kann man weglassen
\item Bindungsstärke nimmt in dieser Reihenfolge ab: \\
$(\neg)$, $(\land,\lor)$, $(\rightarrow, \leftrightarrow)$
\end{itemize}
\item Auch bei Assoziativität \emph{ändern} sich die Formeln, sind aber äquivalent!
\end{itemize}
\subsection{Semantik der Aussagenlogik}
%
Semantik ist die Interpretation der Formeln:
\begin{itemize}
\item Belegung mit Wahrheitswerten
\item Auswertung der Formel\\
Boolesche Funktion: Auswertung unter allen möglichen Belegungen
\end{itemize}
Analogie zu Programmen:
\begin{itemize}
\item \textbf{syntaktisch korrekt} heißt, dass Lexer und Parser fehlerfrei den Term / die Formel fehlerfrei analysieren können, sprich der Code ist Regelkonform
\item \textbf{semantisch korrekt} heißt, dass das richtige Ergebnis geliefert wird (auch logisch korrekt)
\end{itemize}
%
\paragraph{Wahrheitstafel am Beispiel $t = (\neg y \land \neg(x \land) ) \lor z$:}
\ \\ \textbf{WAHRHEITSTAFEL EINFUEGEN}
%TODO
%
\begin{defi}{logische Äquivalenz}
\\Zwei Formeln $s,t$ sind \textbf{logisch äquivalent}, wenn für alle Belegungen $i,j$ $_i = t_j$.\\
Schreibweise: $s \equiv t$

Bsp.:
\[ s = \neg( y \land \neg z ) \equiv t \]
\end{defi}
%
\paragraph{Übung: Sind $s = x \rightarrow y$ und $t = x \land z \rightarrow y \land z$ (logisch) äquivalent?}
\ \\ \textbf{WAHRHEITSTAFEL EINFUEGEN}
%TODO
%
\subsubsection{Regeln der Logik}
\paragraph{Satz: Für beliebige Formeln $s,t,r$ gilt:}
\begin{align*}
\text{Assoziativität}&\begin{cases}
(s \land t) \lor r &\equiv s \land (t \land r)\\
(s \land t) \lor r &\equiv s \land (t \land r)
\end{cases} \\
%
\text{Kommutativität}&\begin{cases}
s \land t &\equiv t \land s\\
s \lor t &\equiv t \lor s
\end{cases} \\
%
\text{Distributivität}&\begin{cases}
s \land (t \lor r) &\equiv s \land t \lor s \land r\\
s \lor t \land r &\equiv (s \lor t) \land (s \lor r)
\end{cases} \\
%
\text{Idempotenz}&\begin{cases}
s \land s &\equiv s \\
s \lor s &\equiv s
\end{cases} \\
%
\text{Neutralität}&\begin{cases}
t \land s &\equiv s \\
f \lor s &\equiv s
\end{cases} \\
%
\text{Dominanz}&\begin{cases}
f \land s &\equiv s \\
t \lor s &\equiv s
\end{cases} \\
%
\text{Absorption}&\begin{cases}
s \land (s \lor t) &\equiv s \\
s \lor (s \land t) &\equiv s
\end{cases} \\
%
\text{Morgan'sche Regel}&\begin{cases}
\neg (s \land t) &\equiv \neg s \lor \neg t\\
\neg (s \lor t) &\equiv \neg s \land \neg t
\end{cases} \\
%
\text{Komplementäreigenschaft}&\begin{cases}
s \land \neg s &\equiv 0 \\
s \lor \neg s &\equiv 1
\end{cases} \\
%
\text{doppelte Negation}&\begin{cases}
\neg (\neg s) &\equiv s
\end{cases} \\
\end{align*}
%
\paragraph{Beispiel:}
\begin{align*}
x_1 \lor (((x_2 'lor x_3)) \land \neg (\neg(\neg x_1(\neg x_1 \lor x_4)))
&\equiv x1 \lor ((x_2 \lor x_3) \land x_1 )  && \mid \text{ Abs., 2x-Neg., Komm.}\\
&\equiv x_1
\end{align*}
%
\paragraph{Übung: }Sind die Operationen $\rightarrow$ und $\leftrightarrow$ assoziativ bzw. kommutativ?
\ \\ \textbf{WAHRHEITSTAFEL EINFUEGEN}
%TODO
%
\paragraph{Gibt es ein neutrales bzw. dominierendes Element bzgl. $\leftrightarrow$?}
\ \\Dominanz: $(0 \leftrightarrow s) \equiv false$\\
Neutralität: $(1 \leftrightarrow s) \equiv s$
%
% Day 1 to Day 2 cut here?
%TODO
%
\paragraph{weitere Äquivalenzen:}
\begin{enumerate}
\item $s \rightarrow t \equiv \neg s \lor t$
\item $s \leftrightarrow t \equiv (\neg s \lor t) \land (s \lor \neg t) \equiv s \land t \lor \neg s \land \neg t$
\item $s \rightarrow t \land r \equiv (s\rightarrow t) \land (s \rightarrow r)$
\item $s \rightarrow t \lor r \equiv (s\rightarrow t) \lor (s \rightarrow r)$
\item $s \land t \rightarrow r \equiv (s \rightarrow r) \lor (t \rightarrow r)$
\item $s \lor t \rightarrow r \equiv (s \rightarrow r) \land (t \rightarrow r)$
\end{enumerate}
%
\begin{defi}
Eine aussagelogische Formel $s$
\begin{itemize}
\item ist \glq erfüllbar\grq , wenn für \emph{(mindestens) eine} Belegung $s$ den Wert $1$ annimmt
\item eine \glq Tautologie\grq\ (oder \glq allgemeingültig\grq ), wenn für \emph{jede} Belegung $s$ $1$ annimmt
\item heißt \glq Kontradiktion\grq , wenn keine Belegung existiert, sodass $s$ $1$ annimmt
\end{itemize}
\end{defi}
%\end{document}