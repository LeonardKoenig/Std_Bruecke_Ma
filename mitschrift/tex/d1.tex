%\documentclass[10pt,a4paper]{article}
%\usepackage[utf8]{inputenc}
%\usepackage[ngerman]{babel}
%\usepackage{amsmath}
%\usepackage{amsfonts}
%\usepackage{amssymb}
%\usepackage{amsthm} % theorems
%\usepackage{cancel} % strikethrough
%\usepackage{tabto} % tabbing
%%\usepackage[left=2cm,right=2cm,top=2cm,bottom=2cm]{geometry}
%\author{Leonard König}
%\title{Mathe-Bruecke_d1}
%\date{2015-09-29}
%\begin{document}
\section{Logik}
\subsection{Aussagenlogik}
Aussage: (formalsprachliches) Gebilde (Satz) mit der Eigenschaft entweder wahr oder falsch zu sein.
%
\begin{itemize}
\item Zweiwertigkeitsprinzip
\item Extentionalitätsprinizip (Wahrheitswert einer zusammengesetzten Aussage ergibt sich aus denen der einzelnen Aussagen)
\end{itemize}
%
\paragraph{Konkatenieren von Aussagen:}
\begin{itemize}
\item Negation (\( \neg \)) \tab{(unär)}
\item Konjunktion (\( \land \)) \tab{(binär)}
\item Disjunktion (\( \lor \)) \tab{(binär)}
\item Implikation (\( \rightarrow \)) \tab{(binär)}
\item Äquivalenz (\( \leftrightarrow \)) \tab{(binär)}
\item ( Antivalenz (\( \oplus \)) ) \tab{(binär)}
\end{itemize}
%
\paragraph{Beispiele und Anwendung}
%
Werte für eine Aussage sind wahr (\( w,t,1 \)) und falsch (\( f,0 \)).
%
\paragraph{Beispiele für Aussagen}
\begin{itemize}
\item \glqq $7$ ist prim\grqq\tab{$1$}
\item \glqq $9$ ist prim\grqq\tab{$0$}
\item \glqq $10$ ist gerade\grqq\tab{$1$}
\item \glqq $7$ ist prim\grqq\tab{$1$}
\item \glqq $7$ ist prim $\lor$ $7$ ist ungerade\grqq\tab{$1$}
\item \glqq $7$ ist prim $\oplus$ $7$ ist ungerade\grqq\tab{$0$}
\item \glqq $\sqrt{2}$ ist rational\grqq\tab{$0$}
\item \glqq Jede gerade Zahl $n\geq 4$\\
kann als Summe von zwei Primzahlen\\
dargestellt werden\grqq\tab{? (Goldbachsche Vermutung)}
\end{itemize}
%
Es muss also \emph{nicht} klar sein, ob die Aussage wahr oder falsch ist!
%
\paragraph{Keine Aussagen}
\begin{itemize}
\item \glqq $x:$ $x$ ist falsch\grqq\tab{Paradoxon}
\item \glqq $n$ ist prim\grqq\tab{Prädikat. Sh. spätere Vorlesungen}
\end{itemize}
%
\textbf{WAHRHEITSTAFEL EINFUEGEN}
%TODO
\subsection{Semantik}
\begin{itemize}
\item \( x + - y \)\tab{syntaktisch inkorrekt}
\item \( \frac{x}{y} \)\tab{syntaktisch korrekt \& semantisch korrekt für \( y \neq 0 \)}
\end{itemize}
%
\paragraph{Definition}
\newtheorem{defi}{Definition}
\begin{defi}[Term der Aussagenlogik]
Gegeben Variablenmenge $V = \{ x,y,z,x_1,x_2, ...\} $
\begin{enumerate}
\item Jede Variable in $V$ und $0,1$ sind \glq Terme\grq\ bzw. Formeln der Aussagenlogik
\item falls $t$ ein Term, dann ist auch ($\neg t$) ein Term\tab{$\mid$ erhöht Rang um 1}
\item sind $s,t$ Terme der Aussagenlogik, \\
dann sind auch $(s \land t)$, $(s \lor t)$, $(s\rightarrow t)$, $(s\leftrightarrow t)$ \\
Terme der Aussagenlogik\tab{$\mid$ neuer Rang: Maximum der beiden \glq Subränge\grq\ $+1$}
\item Alle Terme der Aussagenlogik lassen sich wie beschrieben bilden.
\end{enumerate}
\end{defi}
%
Anmerkungen
\begin{itemize}
\item Wenn wir nur $\neg$, $\land$, $\lor$ verwenden, erzeugen wir die sogenannten \glq Booleschen Terme\grq
\item In der Aussagenlogik sind \glq Term\grq\ und \glq Formel\grq\ synonym.
\item Um auf \emph{Gleichheit} (beachte: \emph{nicht} Äquivalenz!) zu prüfen, können wir den Syntaxbaum nutzen (siehe Abb. ???)
%TODO Abbildungsreferenz
\\ \textbf{SYNTAXBAUM EINFUEGEN}
%TODO
\item Bindungsstärke (zur Klammersetzung):
\begin{itemize}
\item äußere Klammern kann man weglassen
\item Bindungsstärke nimmt in dieser Reihenfolge ab: \\
$(\neg)$, $(\land,\lor)$, $(\rightarrow, \leftrightarrow)$
\end{itemize}
\item Auch bei Assoziativität \emph{ändern} sich die Formeln, sind aber äquivalent!
\end{itemize}
\paragraph{Semantik der Aussagenlogik}
%
Semantik ist die Interpretation der Formeln:
\begin{itemize}
\item Belegung mit Wahrheitswerten
\item Auswertung der Formel\\
Boolesche Funktion: Auswertung unter allen möglichen Belegungen
\end{itemize}
Analogie zu Programmen:
\begin{itemize}
\item \textbf{syntaktisch korrekt} heißt, dass Lexer und Parser fehlerfrei den Term / die Formel fehlerfrei analysieren können, sprich der Code ist Regelkonform
\item \textbf{semantisch korrekt} heißt, dass das richtige Ergebnis geliefert wird (auch logisch korrekt)
\end{itemize}
%
\paragraph{Wahrheitstafel am Beispiel $t = (\neg y \land \neg(x \land) ) \lor z$:}
\ \\ \textbf{WAHRHEITSTAFEL EINFUEGEN}
%TODO
%
\begin{defi}{logische Äquivalenz}
\\Zwei Formeln $s,t$ sind \textbf{logisch äquivalent}, wenn für alle Belegungen $i,j$ $_i = t_j$.\\
Schreibweise: $s \equiv t$

Bsp.:
\[ s = \neg( y \land \neg z ) \equiv t \]
\end{defi}
%
\paragraph{Übung: Sind $s = x \rightarrow y$ und $t = x \land z \rightarrow y \land z$ (logisch) äquivalent?}
\ \\ \textbf{WAHRHEITSTAFEL EINFUEGEN}
%TODO
%
\paragraph{Regeln der Logik}
\paragraph{Satz: Für beliebige Formeln $s,t,r$ gilt:}
\begin{align*}
\text{Assoziativität}&\begin{cases}
(s \land t) \lor r &\equiv s \land (t \land r)\\
(s \land t) \lor r &\equiv s \land (t \land r)
\end{cases} \\
%
\text{Kommutativität}&\begin{cases}
s \land t &\equiv t \land s\\
s \lor t &\equiv t \lor s
\end{cases} \\
%
\text{Distributivität}&\begin{cases}
s \land (t \lor r) &\equiv s \land t \lor s \land r\\
s \lor t \land r &\equiv (s \lor t) \land (s \lor r)
\end{cases} \\
%
\text{Idempotenz}&\begin{cases}
s \land s &\equiv s \\
s \lor s &\equiv s
\end{cases} \\
%
\text{Neutralität}&\begin{cases}
t \land s &\equiv s \\
f \lor s &\equiv s
\end{cases} \\
%
\text{Dominanz}&\begin{cases}
f \land s &\equiv s \\
t \lor s &\equiv s
\end{cases} \\
%
\text{Absorption}&\begin{cases}
s \land (s \lor t) &\equiv s \\
s \lor (s \land t) &\equiv s
\end{cases} \\
%
\text{Morgan'sche Regel}&\begin{cases}
\neg (s \land t) &\equiv \neg s \lor \neg t\\
\neg (s \lor t) &\equiv \neg s \land \neg t
\end{cases} \\
%
\text{Komplementäreigenschaft}&\begin{cases}
s \land \neg s &\equiv 0 \\
s \lor \neg s &\equiv 1
\end{cases} \\
%
\text{doppelte Negation}&\begin{cases}
\neg (\neg s) &\equiv s
\end{cases} \\
\end{align*}
%
\paragraph{Beispiel:}
\begin{align*}
x_1 \lor (((x_2 'lor x_3)) \land \neg (\neg(\neg x_1(\neg x_1 \lor x_4)))
&\equiv x1 \lor ((x_2 \lor x_3) \land x_1 )  && \mid \text{ Abs., 2x-Neg., Komm.}\\
&\equiv x_1
\end{align*}
%
\paragraph{Übung: }Sind die Operationen $\rightarrow$ und $\leftrightarrow$ assoziativ bzw. kommutativ?
\ \\ \textbf{WAHRHEITSTAFEL EINFUEGEN}
%TODO
%
\paragraph{Gibt es ein neutrales bzw. dominierendes Element bzgl. $\leftrightarrow$?}
\ \\Dominanz: $(0 \leftrightarrow s) \equiv false$\\
Neutralität: $(1 \leftrightarrow s) \equiv s$
%
% Day 1 to Day 2 cut here?
%TODO
%
\paragraph{weitere Äquivalenzen:}
\begin{enumerate}
\item $s \rightarrow t \equiv \neg s \lor t$
\item $s \leftrightarrow t \equiv (\neg s \lor t) \land (s \lor \neg t) \equiv s \land t \lor \neg s \land \neg t$
\item $s \rightarrow t \land r \equiv (s\rightarrow t) \land (s \rightarrow r)$
\item $s \rightarrow t \lor r \equiv (s\rightarrow t) \lor (s \rightarrow r)$
\item $s \land t \rightarrow r \equiv (s \rightarrow r) \lor (t \rightarrow r)$
\item $s \lor t \rightarrow r \equiv (s \rightarrow r) \land (t \rightarrow r)$
\end{enumerate}
%
\begin{defi}
Eine aussagelogische Formel $s$
\begin{itemize}
\item ist \glq erfüllbar\grq , wenn für \emph{(mindestens) eine} Belegung $s$ den Wert $1$ annimmt
\item eine \glq Tautologie\grq\ (oder \glq allgemeingültig\grq ), wenn für \emph{jede} Belegung $s$ $1$ annimmt
\item heißt \glq Kontradiktion\grq , wenn keine Belegung existiert, sodass $s$ $1$ annimmt
\end{itemize}
\end{defi}

\subsection{Prädikatenlogik}
\paragraph{Quantoren}
\ \\$ \forall $ Allquantor \\
$ \exists $ Existenzquantor \\

\[
\forall_{x\in U}:P(x) \text{ w.A.},\text{ wenn }P(a) \text{ w.A. f\"ur alle } a \text{ aus } U
\]
\[
\exists_{x\in U}:P(x) \text{ w.A.},\text{ wenn ein }P(a) \text{ existiert, sodass } P(a) \text{ eine w.A.}
\]

\paragraph{Beispiel: Grenzwert-Definition}
\ \\
$(a_n)$ (mit $n\in \mathbb{N}$) hat Grenzwert $a$
\[
\Leftrightarrow \forall_{\varepsilon > 0} \, \exists_{n_0 \in \mathbb{N}} \, \forall_{n \geq n_0}: |a_n - a| < \varepsilon
\]

\paragraph{Definition (Pr\"adikat):}
\ \\
\glqq Ein logisches Pr\"adikat ist ein Ausdruck, der eine oder mehrere Variablen enth\"alt, sodass bei Belegung aller Variablen mit Objekten aus dem festgelegten Individuenbereich $B\setminus U$ eine Aussage entsteht.\grqq \\
\ \\
Beispiel:\\
\[
U=\mathbb{N}=\{0,1,2,3,\ldots\}
\]
$P(x)$ : \glq $x$ ist prim\grq \\
$R(x,y) = $\glq$x \leq y$\grq \\
$Q(x,y) = $\glq$x=y+2$\grq

\paragraph{\"Aquivalenzen bei Pr\"adikaten}
\begin{enumerate}
\item $\neg \forall_x : P(x) \equiv \exists_x: \neg P(x)$
\item $\neg \exists_x : P(x) \equiv \forall_x: \neg P(x)$
\item $\forall_x : P(x) \land \forall_x : Q(x) \equiv \forall : P(x) \land Q(x)$
\item $\exists_x : P(x) \lor \exists_x : Q(x) \equiv \exists_x : P(x) \lor Q(x)$
\item $\forall_{x,y} : R(x,y) \equiv \forall_{y,x} : R(x,y)$
\item $\exists_{x,y} : R(x,y) \equiv \exists_{y,x} : R(x,y)$
\item[$\perp$] $\forall_x : P(x) \lor \forall_x:Q(x) \equiv \forall_x : P(x) \lor Q(x)$
\item[$\perp$] $\exists_x : P(x) \land \exists : Q(x) \equiv \exists_x : P(x) \land Q(x)$
\item[$\perp$] $\forall_x \exists_y : R(x,y) \equiv \exists_y \forall_x : R(x,y)$
\end{enumerate}

\subsection{Beweistechniken}

\paragraph{Basiswissen:}
\ \\
\begin{itemize}
\item $\mathbb{N} = \{0,1,2,3,\ldots\}$ nat\"urliche Zahlen
\item $\mathbb{N}^{+} = \mathbb{N}\setminus\{0\}$
\item Addition und Multiplikation sind kommutativ, assoziativ und distributiv
\item $n\in\mathbb{N}$ ist durch $d\in\mathbb{N}^+$ teilbar (auch: \glqq d teilt u\grqq ), wenn ein $k\in\mathbb{N}$ existiert, sodass $n=k\cdot d$\\
Notation: $d|n$
\item $p\geq 2$ aus $\mathbb{N}$ ist prim, wenn sie nur durch $p$ und $1$ teilbar sind.
\item Jedes $n\geq 2$ besitzt eine \emph{eindeutige} Zerlegung in Primfaktoren, d.h.
\[
n = p_1 \cdot p_2 \cdot \ldots \cdot p_k
\]
mit $p_k$ Primfaktoren. Dies gilt insbesondere f\"ur $n=n$
\item $n\in\mathbb{N}$ ist gerade, wenn $n$ durch $2$ teilbar ist, d.h. $n=2k$ für ein $k\in\mathbb{N}$\\
$n$ ist ungerade, wenn ein solches k nicht existiert, bzw. $n=2k'+1$
\item $\forall n\in\mathbb{N},d\in\mathbb{N}^{+}$ gibt es eindeutige Werte $q,r\in\mathbb{N}$ mit $n=q\cdot d+r$ und $0\leq r< d$ mit $r=n\bmod d$
\end{itemize}

\subsubsection{Direkter Beweis}
Ein direkter Beweis hat die Form $p\rightarrow q$ mit Prämisse $p$ und Behauptung $q$. Man baut eine \glqq logische Kette\grqq , nach folgendem Schema:
\[
p=p_1\rightarrow p_2 \rightarrow p_3 \rightarrow \ldots \rightarrow p_k=q
\]
Wobei alle $p_n\rightarrow p_{n+1}$ möglichst triviale Folgerungen sind.

\paragraph{Beispiel:}Wenn $\underbrace{l|m \land m|n}_{p}$, dann $\underbrace{l|n}_{q}$.
\begin{proof}
\begin{align}
&l|m \land m|n \\
\Rightarrow &\exists_{k\in\mathbb{N}}: m=k\cdot l \land \exists_{j\in\mathbb{N}}: n=j\cdot m \\
\Rightarrow &\exists_{k,j\in\mathbb{N}}: m=k\cdot l \land n=j\cdot m \\
\Rightarrow &\exists_{k,j\in\mathbb{N}}: n=j(k\cdot l) \\
\Rightarrow &\exists_{k,j\in\mathbb{N}}: n=(j\cdot k)l && \mid k'=j\cdot k \\
\Rightarrow &\exists_{k,j\in\mathbb{N}}: n=k' \cdot l \\
\Rightarrow &l|n
\end{align}
\end{proof}

\subsubsection{Beweis durch Fallunterscheidung}
Ein Beweis der Form $p\rightarrow q \equiv \underbrace{(p\land r\rightarrow q)}_\text{Fall 1} \land \underbrace{(p\land\neg r \rightarrow q)}_\text{Fall 2}$ ist ein Beweis durch Fallunterscheidung, da wir die Fälle $r$ und $\neg r$ unterscheidet betrachten und für beide einen Teilbeweis führen.\\
Beispiel: sh. Beispiel zur Kontraposition.
\subsubsection{Indirekter Beweis}
\begin{itemize}
\item Kontra\emph{position} (Annahme der Gegenposition) \\
$p\rightarrow q \equiv \neg q \rightarrow\neg p$
\item Kontra\emph{diktion} (Widerspruch) \\
$p\rightarrow q \equiv p \land\neg q \rightarrow 0$
\end{itemize}

\paragraph{Beispiel}
$k|m\land k\nmid n \Rightarrow m\nmid n$
\paragraph{Beispiel (Widerspruch):}
\begin{proof}
\begin{align*}
&k|m\land k\nmid n \land m\mid n \\
\Rightarrow &k|m\land k\nmid n \land m\mid n && \mid k|m \land m|n \Rightarrow k|n \\
\Rightarrow &k|n \land k\nmid n \\
\Rightarrow &\bot
\end{align*}
\end{proof}

\paragraph{Beispiel (Kontraposition):}
\begin{proof}
\[
m|n \Rightarrow k\nmid m \lor k|n
\]
\begin{itemize}
\item Fall 1 ($k\nmid m$): w.A. nach $\neg p$
\item Fall 2 ($k\mid m$): $m|n \land k|m \Rightarrow k|n$, w.A. nach $\neg p$
\end{itemize}
\end{proof}

\subsection{Induktion}
\subsubsection{Einschub: Peano-Axiome der natürlichen Zahlen}
\begin{itemize}
\item $0\in\mathbb{N}$
\item fúr jedes $n\in\mathbb{N}$ gibt es einen \emph{eindeutigen} Nachfolger $s(n)\in\mathbb{N}$
\item verschiedene natürliche Zahlen haben verschiedene Nachfolger:\\
$n\neq n' \Rightarrow s(n) \neq s(n')$
\item $0$ ist kein Nachfolger
\item $\mathbb{N}$ ist die \emph{kleinste Menge, die diese Bedingungen erfüllt.}
\end{itemize}


\paragraph{Konsequenz: Rekursion}Man kann eine Funktion $f$ auf der Menge der natürlichen Zahlen definieren, indem man
\begin{itemize}
\item $f(0)$ festlegt und
\item $f(s(n))$ auf $f(n)$ zurückführt
\end{itemize}
Bspw.: $f:\mathbb{N}\rightarrow\{0,1\}$ mit
\begin{itemize}
\item $f(0)=0$
\item $f(s(n))=\neg f(n)$
\end{itemize}
%TODO
\textbf{TABELLE EINFUEGEN!!!}
% Auf Seite 14 oben

Insbesondere aber auch die Addition bzgl. $\mathbb{N}$:
\begin{align*}
f_n:\mathbb{N}\rightarrow\mathbb{N}\\
n\mapsto m+n
\end{align*}
\begin{enumerate}
\item $f_m(0)=m$
\item $f_m(s(n)) = s(f_m(n))$
\end{enumerate}

\paragraph{Am Beispiel $4+2$:}
\[
f_4(2)=f_4(s(1))=s(f_4(1))=s(f_4(s(0)))=s(s(f_4(0)))=s(s(4))=s(5)=6
\]
\subsubsection{Rekursion als Beweismethode: Induktion}
\paragraph{Idee:}Ein Prädikat $P(n)$ wobei $n\in\mathbb{N}$ ist wahr für alle $n\in\mathbb{N}$, wenn
\begin{enumerate}
\item $P(0)$ wahr ist und
\item für beliebige $n$ aus $P(n)$ $P(n+1)$ folgt (\glqq Dominoprinzip\grqq )
\end{enumerate}

(1) nennt man dabei \textbf{Induktionsanfang} oder \textbf{Induktionsverankerung} und (2) den \textbf{Induktionsschritt}, wobei man letzteren in die Gültigkeit von $P(n)$ als \textbf{Induktionsvoraussetzung} und die der \textbf{Induktionsbehauptung} $P(n+1)$ unterteilen kann.

\paragraph{Beispiel durch vollständige Induktion nach $n$:}$6|n^3 -n,\, \forall_{n\in\mathbb{N}}$\\
\begin{proof}
\ \\
\textbf{IA.:} $n=0$: $0^3-0=0$, w.A. \\
\textbf{IV.:} $6|n^3 -n$ w.A. für ein $n$ \\
\textbf{IB.:} aus $6|n^3 -n \rightarrow 6|(n-1)^3 -(n-1)$ \\
\textbf{IS.:}
\begin{align*}
&(n+1)^3 - (n+1) \\
= &n^3+3n^2+3n+1-(n+1) \\
= &(n^3-n) + 3n^2 + 3n && |k=\frac{n^3-n}{6} \text{ nach IV.}\\
= &6k +3(n^2+n) \\
= &6k + 3(n(n+1)) && |\text{Da $n$ oder $n+1$ gerade: } l=\frac{n(n+1)}{2} \\
= &6k+3\cdot 2l \\
= &6(k+l)
\end{align*}
\end{proof}


%\end{document}